   % /--------------------------------------------\
   % | API-Dokumentation für einige Java-Packages |
   % |    (genaueres siehe doku-main.tex).        |
   % | LaTeX-Ausgabe erstellt von 'ltxdoclet'.    |
   % | Dieses Programm stammt von Paul Ebermann.  |
   % \--------------------------------------------/

   % Api-Dokumentation für Klasse ontologyFramework.OFEventManagement.OFLogicalEventManagement.OFEventAggregation (noch nicht fertig). 
\section[OFEventAggregation]{Klasse \ltdHypertarget{ontologyFramework.OFEventManagement.OFLogicalEventManagement.OFEventAggregation-class}{ontologyFramework.OFEventManagement.OFLogicalEventManagement.OFEventAggregation}}\label{ontologyFramework.OFEventManagement.OFLogicalEventManagement.OFEventAggregation-class}
\subsection{Übersicht}
This class represent the event linked to every individual belong to the ontological class \verb!"Event"!.
 \par 
 It is based on a boolean parameterized expression given as a String. And a list of parameters that can 
 be added or removed. All parameters are expressed in therms of instances 
 of \noprint{@link:OFEventDefinition}\texttt{\hyperlink{ontologyFramework.OFEventManagement.OFLogicalEventManagement.OFEventDefinition-class}{OFEventDefinition}} and they must tagged with the name used in the expression.
\begin{description}
\item[@author] 
Buoncomapgni Luca
\item[@version] 
1.0
\end{description}
\subsection{Inhaltsverzeichnis}
\subsection{Konstruktoren}
\begin{description}
\item[{\ltdHypertarget{ontologyFramework.OFEventManagement.OFLogicalEventManagement.OFEventAggregation(java.lang.String)}{OFEventAggregation}\label{ontologyFramework.OFEventManagement.OFLogicalEventManagement.OFEventAggregation(java.lang.String)}}]
~ Create new event with a specific boolean low as a String. This will be processed by
 the library \noprint{@link:MVEL}\texttt{MVEL} on runtime and must return always a boolean value.
\begin{description}
\item[Parameter] ~
\begin{description}
\item[aggregationLow]
parameterized logical relation
\end{description}
\end{description}
\end{description}
\subsection{Methoden}
\begin{description}
\item[{\ltdHypertarget{ontologyFramework.OFEventManagement.OFLogicalEventManagement.OFEventAggregation.addParameter(java.lang.String,ontologyFramework.OFEventManagement.OFLogicalEventManagement.OFEventDefinition)}{addParameter}\label{ontologyFramework.OFEventManagement.OFLogicalEventManagement.OFEventAggregation.addParameter(java.lang.String,ontologyFramework.OFEventManagement.OFLogicalEventManagement.OFEventDefinition)}}]
~ Add a new parameter to the definition of this event tagged with the variable
 name used to define the aggregation low. This class must contains one OFEventDefinition 
 for each name used in the \verb!String aggregationLow!. 
 (ex: \verb!"!r1 && r2"!) where r1 and r2 are variables.
\begin{description}
\item[Parameter] ~
\begin{description}
\item[varName]
the name of the variable used in the String \verb!aggregationLow!
\item[eventDef]
initialized definition of the parameter
\end{description}
\end{description}
\item[{\ltdHypertarget{ontologyFramework.OFEventManagement.OFLogicalEventManagement.OFEventAggregation.removeParameterMap(java.lang.String)}{removeParameterMap}\label{ontologyFramework.OFEventManagement.OFLogicalEventManagement.OFEventAggregation.removeParameterMap(java.lang.String)}}]
~ Remove a parameter from definition of this event.
\begin{description}
\item[Parameter] ~
\begin{description}
\item[varName]
the name of the variable used in the String \verb!aggregationLow!
\end{description}
\end{description}
\item[{\ltdHypertarget{ontologyFramework.OFEventManagement.OFLogicalEventManagement.OFEventAggregation.clearParameterMap()}{clearParameterMap}\label{ontologyFramework.OFEventManagement.OFLogicalEventManagement.OFEventAggregation.clearParameterMap()}}]
~ Remove all the parameters from definition of this event.
\item[{\ltdHypertarget{ontologyFramework.OFEventManagement.OFLogicalEventManagement.OFEventAggregation.compute(ontologyFramework.OFRunning.OFInvokingManager.OFBuildedListInvoker)}{compute}\label{ontologyFramework.OFEventManagement.OFLogicalEventManagement.OFEventAggregation.compute(ontologyFramework.OFRunning.OFInvokingManager.OFBuildedListInvoker)}}]
~ Compute event result. It goes for all the parameter added to 
 this class and calls \verb!ofEventDefinition.compute( invoker)!. 
 Finally, the returning boolean value is used to compute the aggregation low.
\begin{description}
\item[Parameter] ~
\begin{description}
\item[invoker]
builded list of class during startup used by \noprint{@link:OFEventInterface#evaluateEvent(java.util.List, OFBuildedListInvoker)}\texttt{\hyperlink{ontologyFramework.OFEventManagement.OFLogicalEventManagement.OFEventInterface.evaluateEvent(java.util.List<ontologyFramework.OFEventManagement.EventComputedData>,ontologyFramework.OFRunning.OFInvokingManager.OFBuildedListInvoker)}{evaluateEvent}}
\end{description}
\item[Rückgabewert] 
true if the event occurs in this moment
\end{description}
\end{description}
