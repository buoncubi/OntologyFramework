   % /--------------------------------------------\
   % | API-Dokumentation für einige Java-Packages |
   % |    (genaueres siehe doku-main.tex).        |
   % | LaTeX-Ausgabe erstellt von 'ltxdoclet'.    |
   % | Dieses Programm stammt von Paul Ebermann.  |
   % \--------------------------------------------/

   % Api-Dokumentation für Klasse ontologyFramework.OFEventManagement.OFLogicalEventManagement.OFEventImplementation.IsInClass (noch nicht fertig). 
\section[IsInClass]{Klasse \ltdHypertarget{ontologyFramework.OFEventManagement.OFLogicalEventManagement.OFEventImplementation.IsInClass-class}{ontologyFramework.OFEventManagement.OFLogicalEventManagement.OFEventImplementation.IsInClass}}\label{ontologyFramework.OFEventManagement.OFLogicalEventManagement.OFEventImplementation.IsInClass-class}
\subsection{Übersicht}
This class implement the event that takes as input : \verb!...( OWLNamedIndividual ind, OWLClass cl)!.
 Which return true if the individual belongs to the class and false otherwise. In the ontology 
 an event must be defined which belongs to the class \verb!"OFEvent"! thus has the properties:
 {\ttfamily
\mbox{ }\mbox{}\newline
\mbox{ }	\verb!implementsOFEventName "ontologyFramework.OFEventManagement.OFEventImplementation.OFEventProcedure.IsInClass"^^string!\mbox{}\newline
\mbox{ }\mbox{ }\verb!&!\mbox{}\newline
\mbox{ }\mbox{ }\verb!hasEvent definition 
  		"in:ontologyFramework.OFEventManagement.OFEventParameter.
			 ?a @ontoName(S1-3) Exception.AsOWLClass
			 ?b @ontoName exc.AsOWLIndividual
			 !r IsInClass(?b ?a)"^^strign!\mbox{}\newline
\mbox{ }}

 So, \noprint{@link:#isCorrectInput(List)}\texttt{\hyperlink{ontologyFramework.OFEventManagement.OFLogicalEventManagement.OFEventImplementation.IsInClass.isCorrectInput(java.util.List<ontologyFramework.OFEventManagement.EventComputedData>)}{isCorrectInput}} return true if \verb!ts.get(0).getParameter() instanceof OWLNamedIndividual!
 and \verb!inputs.get( 1).getParameter() instanceof OWLClass!, \verb!inputs.get(1).getOntoRef() != null! and \verb!inputs.get(0).getOntoRef() != null!, are true.
 The \noprint{@link:#evaluateEvent(List, OFBuildedListInvoker)}\texttt{\hyperlink{ontologyFramework.OFEventManagement.OFLogicalEventManagement.OFEventImplementation.IsInClass.evaluateEvent(java.util.List<ontologyFramework.OFEventManagement.EventComputedData>,ontologyFramework.OFRunning.OFInvokingManager.OFBuildedListInvoker)}{evaluateEvent}} just ask to the reasoner of the ontology named
 \verb!"ontoName(S1-3)"! if in the class \textquotedbl Exception\textquotedbl  exist an individual called \textquotedbl exc\textquotedbl  and propagates the
 answare.
\begin{description}
\item[@author] 
Buoncomapgni Luca
\item[@version] 
1.0
\end{description}
\subsection{Inhaltsverzeichnis}
\subsection{Konstruktoren}
\begin{description}
\item[{\ltdHypertarget{ontologyFramework.OFEventManagement.OFLogicalEventManagement.OFEventImplementation.IsInClass()}{IsInClass}\label{ontologyFramework.OFEventManagement.OFLogicalEventManagement.OFEventImplementation.IsInClass()}}]
~ 
\end{description}
\subsection{Methoden}
\begin{description}
\item[{\ltdHypertarget{ontologyFramework.OFEventManagement.OFLogicalEventManagement.OFEventImplementation.IsInClass.isCorrectInput(java.util.List<ontologyFramework.OFEventManagement.EventComputedData>)}{isCorrectInput}\label{ontologyFramework.OFEventManagement.OFLogicalEventManagement.OFEventImplementation.IsInClass.isCorrectInput(java.util.List<ontologyFramework.OFEventManagement.EventComputedData>)}}]
~ 
\item[{\ltdHypertarget{ontologyFramework.OFEventManagement.OFLogicalEventManagement.OFEventImplementation.IsInClass.evaluateEvent(java.util.List<ontologyFramework.OFEventManagement.EventComputedData>,ontologyFramework.OFRunning.OFInvokingManager.OFBuildedListInvoker)}{evaluateEvent}\label{ontologyFramework.OFEventManagement.OFLogicalEventManagement.OFEventImplementation.IsInClass.evaluateEvent(java.util.List<ontologyFramework.OFEventManagement.EventComputedData>,ontologyFramework.OFRunning.OFInvokingManager.OFBuildedListInvoker)}}]
~ 
\end{description}
