   % /--------------------------------------------\
   % | API-Dokumentation für einige Java-Packages |
   % |    (genaueres siehe doku-main.tex).        |
   % | LaTeX-Ausgabe erstellt von 'ltxdoclet'.    |
   % | Dieses Programm stammt von Paul Ebermann.  |
   % \--------------------------------------------/

   % Api-Dokumentation für Klasse ontologyFramework.OFEventManagement.OFLogicalEventManagement.OFEventBuilder (noch nicht fertig). 
\section[OFEventBuilder]{Klasse \ltdHypertarget{ontologyFramework.OFEventManagement.OFLogicalEventManagement.OFEventBuilder-class}{ontologyFramework.OFEventManagement.OFLogicalEventManagement.OFEventBuilder}}\label{ontologyFramework.OFEventManagement.OFLogicalEventManagement.OFEventBuilder-class}
\subsection{Übersicht}
This class, as all the class that implements \noprint{@link:OFBuilderInterface}\texttt{\hyperlink{ontologyFramework.OFRunning.OFInitialising.OFBuilderInterface<T>-class}{OFBuilderInterface}} has the proposes to initialize
 classes to be used during system evolution. In this case its initializes Events,
 in particular the classes: \noprint{@link:OFEventParameterDefinition}\texttt{\hyperlink{ontologyFramework.OFEventManagement.OFEventParameterDefinition-class}{OFEventParameterDefinition}}, \noprint{@link:OFEventDefinition}\texttt{\hyperlink{ontologyFramework.OFEventManagement.OFLogicalEventManagement.OFEventDefinition-class}{OFEventDefinition}} and \noprint{@link:OFEventAggregation}\texttt{\hyperlink{ontologyFramework.OFEventManagement.OFLogicalEventManagement.OFEventAggregation-class}{OFEventAggregation}}. 
 \par 
 A call to \noprint{@link:#buildInfo(String[], OWLReferences, OFBuildedListInvoker)}\texttt{\hyperlink{ontologyFramework.OFEventManagement.OFLogicalEventManagement.OFEventBuilder.buildInfo(java.lang.String[],ontologyFramework.OFContextManagement.OWLReferences,ontologyFramework.OFRunning.OFInvokingManager.OFBuildedListInvoker)}{buildInfo}} causes the reset of the
 initialized classes Map, then all the individual inside the ontological class, named \verb!keyWord[ 0]!
 (by default: \textquotedbl Event\textquotedbl ) are processed. Where, The definition of this class must be:
 	\verb!(hasTypeEventParameter min 1 string) and (hasTypeEventComputeLow exactly 1 string)! 
 \par 
 For all of them it retrieves the computational low (ex: \textquotedbl r1 \&\& r2\textquotedbl ) as a string and creates a 
 new \noprint{@link:OFEventAggregation}\texttt{\hyperlink{ontologyFramework.OFEventManagement.OFLogicalEventManagement.OFEventAggregation-class}{OFEventAggregation}}. Than, it gets the value of the data property 
 named \verb!keyWord[ 1]!$\}$ (by default: \textquotedbl hasTypeEventParameter\textquotedbl ) and 
 it parse the incoming value (for example: \textquotedbl r1 = OFEventProcedure\_IndName\textquotedbl ) using the symbol 
 \noprint{@link:#ASSEGNATION_symb}\texttt{\hyperlink{ontologyFramework.OFEventManagement.OFLogicalEventManagement.OFEventBuilder.ASSEGNATION_symb}{ASSEGNATION_symb}}; by default <<@value:#ASSEGNATION_symb>>. 
 Note that the parameter is discarded if the parse has no two token. 
 Then for each parameter belong to the individual a new 
 \noprint{@link:OFEventDefinition}\texttt{\hyperlink{ontologyFramework.OFEventManagement.OFLogicalEventManagement.OFEventDefinition-class}{OFEventDefinition}} is created with the full identify java class which belongs to the individual 
 \textquotedbl OFEventProcedure\_IndName\textquotedbl , Name retrieved from the value of the \verb!keyWord[ 2]!(by default: implements OFEventName).  
 Finally, it gets parameters trough ontological individual linked by the object property named: \verb!keyWord[ 3]! 
 (by Default: \textquotedbl hasEventDefinition\textquotedbl ). Parameters are  added to the Event Definition thanks a parsing 
 Mechanism of the data type value belong to the data property: \verb!keyWord[ 4]! 
 (by default \textquotedbl hasTypeEventDefinition\textquotedbl ).   
 \par 
 The call to the method \noprint{@link:#getInitialisedClasses()}\texttt{\hyperlink{ontologyFramework.OFEventManagement.OFLogicalEventManagement.OFEventBuilder.getInitialisedClasses()}{getInitialisedClasses}} after called \noprint{@link:#initializeDefinition(OWLNamedIndividual, OFEventDefinition, OWLReferences)}\texttt{initializeDefinition}
 returns a \verb!HashMap<String,! \noprint{@link:OFEventAggregation}\texttt{\hyperlink{ontologyFramework.OFEventManagement.OFLogicalEventManagement.OFEventAggregation-class}{OFEventAggregation}}\verb!>! where, keys are the names of the individuals
 belong to the class named \verb!keyWord[ 0]!. While the values are the classes which represent and
 allow to compute all the Events available during the calling of \noprint{@link:#initializeDefinition(OWLNamedIndividual, OFEventDefinition, OWLReferences)}\texttt{initializeDefinition}
\begin{description}
\item[@author] 
Buoncomapgni Luca
\item[@version] 
1.0
\end{description}
\subsection{Inhaltsverzeichnis}
\subsection{Variablen}
\begin{description}
\item[{\ltdHypertarget{ontologyFramework.OFEventManagement.OFLogicalEventManagement.OFEventBuilder.ASSEGNATION_symb}{ASSEGNATION\_symb}\label{ontologyFramework.OFEventManagement.OFLogicalEventManagement.OFEventBuilder.ASSEGNATION_symb}}]
~ Symbol for divide parameters and accept tokens, used only in Event aggregation
\item[{\ltdHypertarget{ontologyFramework.OFEventManagement.OFLogicalEventManagement.OFEventBuilder.ENDLine_symb}{ENDLine\_symb}\label{ontologyFramework.OFEventManagement.OFLogicalEventManagement.OFEventBuilder.ENDLine_symb}}]
~ System symbol to end a line, it represents the end of a command
\item[{\ltdHypertarget{ontologyFramework.OFEventManagement.OFLogicalEventManagement.OFEventBuilder.ASSEGNATIONPARAMETER_symb}{ASSEGNATIONPARAMETER\_symb}\label{ontologyFramework.OFEventManagement.OFLogicalEventManagement.OFEventBuilder.ASSEGNATIONPARAMETER_symb}}]
~ Symbol to assign parameters to a variable. It represents an assegnation during Parameter definition
\item[{\ltdHypertarget{ontologyFramework.OFEventManagement.OFLogicalEventManagement.OFEventBuilder.VARIABLE_symb}{VARIABLE\_symb}\label{ontologyFramework.OFEventManagement.OFLogicalEventManagement.OFEventBuilder.VARIABLE_symb}}]
~ Symbol which identify that the word used before than the next \noprint{@link:#SPLIT_symb}\texttt{\hyperlink{ontologyFramework.OFEventManagement.OFLogicalEventManagement.OFEventBuilder.SPLIT_symb}{SPLIT_symb}} 
 is a local variable.
\item[{\ltdHypertarget{ontologyFramework.OFEventManagement.OFLogicalEventManagement.OFEventBuilder.RETURN_symb}{RETURN\_symb}\label{ontologyFramework.OFEventManagement.OFLogicalEventManagement.OFEventBuilder.RETURN_symb}}]
~ Symbol which identify that the word used before than the next \noprint{@link:#SPLIT_symb}\texttt{\hyperlink{ontologyFramework.OFEventManagement.OFLogicalEventManagement.OFEventBuilder.SPLIT_symb}{SPLIT_symb}} 
 is a the actual event instruction and no more a parameter.
\item[{\ltdHypertarget{ontologyFramework.OFEventManagement.OFLogicalEventManagement.OFEventBuilder.ATONTOLOGY_symb}{ATONTOLOGY\_symb}\label{ontologyFramework.OFEventManagement.OFLogicalEventManagement.OFEventBuilder.ATONTOLOGY_symb}}]
~ It can be used after the declaration of a variable and identify the \noprint{@link:OWLReferences}\texttt{\hyperlink{ontologyFramework.OFContextManagement.OWLReferences-class}{OWLReferences}}
 name i in whihc the parameter must be retrieved. If it is not specified than the
 corrent ontology is considered.
\item[{\ltdHypertarget{ontologyFramework.OFEventManagement.OFLogicalEventManagement.OFEventBuilder.COMMAND_symb}{COMMAND\_symb}\label{ontologyFramework.OFEventManagement.OFLogicalEventManagement.OFEventBuilder.COMMAND_symb}}]
~ Symbol used the decide chains of computation to retrieve parameter, where 
 they must to contains \noprint{@link:#SPLIT_symb}\texttt{\hyperlink{ontologyFramework.OFEventManagement.OFLogicalEventManagement.OFEventBuilder.SPLIT_symb}{SPLIT_symb}}. An example is:
 \verb!name.AsInteger.AsIntegerOWLDataProperty!, equivalent to write
 \verb!AsIntegerOWLDataProperty( AsInteger( name.toString()))!; in other languages.
\item[{\ltdHypertarget{ontologyFramework.OFEventManagement.OFLogicalEventManagement.OFEventBuilder.NULL_symb}{NULL\_symb}\label{ontologyFramework.OFEventManagement.OFLogicalEventManagement.OFEventBuilder.NULL_symb}}]
~ Intercepts whenever null value should be given as input to computer parameter. It can
 be used only as a first element of parameter computation.
\item[{\ltdHypertarget{ontologyFramework.OFEventManagement.OFLogicalEventManagement.OFEventBuilder.STARTPARAMETR_symb}{STARTPARAMETR\_symb}\label{ontologyFramework.OFEventManagement.OFLogicalEventManagement.OFEventBuilder.STARTPARAMETR_symb}}]
~ It defines the starting point in which parameter are used inside the event definition.
 It must be used in the returning line defined by \noprint{@link:#RETURN_symb}\texttt{\hyperlink{ontologyFramework.OFEventManagement.OFLogicalEventManagement.OFEventBuilder.RETURN_symb}{RETURN_symb}}. Between this two symbol 
 no check of the name is provided.
\item[{\ltdHypertarget{ontologyFramework.OFEventManagement.OFLogicalEventManagement.OFEventBuilder.ENDPARAMETER_symb}{ENDPARAMETER\_symb}\label{ontologyFramework.OFEventManagement.OFLogicalEventManagement.OFEventBuilder.ENDPARAMETER_symb}}]
~ It defines the starting point in which parameter are used inside the event definition.
\item[{\ltdHypertarget{ontologyFramework.OFEventManagement.OFLogicalEventManagement.OFEventBuilder.SPLIT_symb}{SPLIT\_symb}\label{ontologyFramework.OFEventManagement.OFLogicalEventManagement.OFEventBuilder.SPLIT_symb}}]
~ Symbol used compute tokens of every lines.
\item[{\ltdHypertarget{ontologyFramework.OFEventManagement.OFLogicalEventManagement.OFEventBuilder.IMPORT_symb}{IMPORT\_symb}\label{ontologyFramework.OFEventManagement.OFLogicalEventManagement.OFEventBuilder.IMPORT_symb}}]
~ Symbol used to define the full identify package in which all the computational
 method to compute parameters are located. This string is added to the name of the name of
 the procedure ( ex: \textquotedbl in: java.package.\textquotedbl  + \textquotedbl AsIntegerOWLDataProperty).
\end{description}
\subsection{Konstruktoren}
\begin{description}
\item[{\ltdHypertarget{ontologyFramework.OFEventManagement.OFLogicalEventManagement.OFEventBuilder()}{OFEventBuilder}\label{ontologyFramework.OFEventManagement.OFLogicalEventManagement.OFEventBuilder()}}]
~ 
\end{description}
\subsection{Methoden}
\begin{description}
\item[{\ltdHypertarget{ontologyFramework.OFEventManagement.OFLogicalEventManagement.OFEventBuilder.buildInfo(java.lang.String[],ontologyFramework.OFContextManagement.OWLReferences,ontologyFramework.OFRunning.OFInvokingManager.OFBuildedListInvoker)}{buildInfo}\label{ontologyFramework.OFEventManagement.OFLogicalEventManagement.OFEventBuilder.buildInfo(java.lang.String[],ontologyFramework.OFContextManagement.OWLReferences,ontologyFramework.OFRunning.OFInvokingManager.OFBuildedListInvoker)}}]
~ 
\item[{\ltdHypertarget{ontologyFramework.OFEventManagement.OFLogicalEventManagement.OFEventBuilder.getInitialisedClasses()}{getInitialisedClasses}\label{ontologyFramework.OFEventManagement.OFLogicalEventManagement.OFEventBuilder.getInitialisedClasses()}}]
~ 
\item[{\ltdHypertarget{ontologyFramework.OFEventManagement.OFLogicalEventManagement.OFEventBuilder.getText(java.util.Set<org.semanticweb.owlapi.model.OWLLiteral>)}{getText}\label{ontologyFramework.OFEventManagement.OFLogicalEventManagement.OFEventBuilder.getText(java.util.Set<org.semanticweb.owlapi.model.OWLLiteral>)}}]
~ 
\end{description}
