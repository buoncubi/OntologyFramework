   % /--------------------------------------------\
   % | API-Dokumentation für einige Java-Packages |
   % |    (genaueres siehe doku-main.tex).        |
   % | LaTeX-Ausgabe erstellt von 'ltxdoclet'.    |
   % | Dieses Programm stammt von Paul Ebermann.  |
   % \--------------------------------------------/

   % Api-Dokumentation für Klasse ontologyFramework.OFEventManagement.OFLogicalEventManagement.OFEventInterface (noch nicht fertig). 
\section[OFEventInterface]{Interface \ltdHypertarget{ontologyFramework.OFEventManagement.OFLogicalEventManagement.OFEventInterface-class}{ontologyFramework.OFEventManagement.OFLogicalEventManagement.OFEventInterface}}\label{ontologyFramework.OFEventManagement.OFLogicalEventManagement.OFEventInterface-class}
\subsection{Übersicht}
This interface is used to define the event procedure. It is called
 from \noprint{@link:OFEventDefinition#compute(OFBuildedListInvoker)}\texttt{\hyperlink{ontologyFramework.OFEventManagement.OFLogicalEventManagement.OFEventDefinition.compute(ontologyFramework.OFRunning.OFInvokingManager.OFBuildedListInvoker)}{compute}} which calls
 \noprint{@link:#isCorrectInput(List)}\texttt{\hyperlink{ontologyFramework.OFEventManagement.OFLogicalEventManagement.OFEventInterface.isCorrectInput(java.util.List<ontologyFramework.OFEventManagement.EventComputedData>)}{isCorrectInput}} first and, if the result is true it calls
 \noprint{@link:#evaluateEvent(List, OFBuildedListInvoker)}\texttt{\hyperlink{ontologyFramework.OFEventManagement.OFLogicalEventManagement.OFEventInterface.evaluateEvent(java.util.List<ontologyFramework.OFEventManagement.EventComputedData>,ontologyFramework.OFRunning.OFInvokingManager.OFBuildedListInvoker)}{evaluateEvent}} and propagate the result
 to the \noprint{@link:OFEventAggregation#compute(OFBuildedListInvoker)}\texttt{\hyperlink{ontologyFramework.OFEventManagement.OFLogicalEventManagement.OFEventAggregation.compute(ontologyFramework.OFRunning.OFInvokingManager.OFBuildedListInvoker)}{compute}}.
\begin{description}
\item[@author] 
Buoncomapgni Luca
\item[@version] 
1.0
\end{description}
\subsection{Inhaltsverzeichnis}
\subsection{Methoden}
\begin{description}
\item[{\ltdHypertarget{ontologyFramework.OFEventManagement.OFLogicalEventManagement.OFEventInterface.isCorrectInput(java.util.List<ontologyFramework.OFEventManagement.EventComputedData>)}{isCorrectInput}\label{ontologyFramework.OFEventManagement.OFLogicalEventManagement.OFEventInterface.isCorrectInput(java.util.List<ontologyFramework.OFEventManagement.EventComputedData>)}}]
~ sviluppated with safety pourposes it is called to check
 if the type of parameter in inputs are correct. If this return
 false the event result of \noprint{@link:#evaluateEvent(List, OFBuildedListInvoker)}\texttt{\hyperlink{ontologyFramework.OFEventManagement.OFLogicalEventManagement.OFEventInterface.evaluateEvent(java.util.List<ontologyFramework.OFEventManagement.EventComputedData>,ontologyFramework.OFRunning.OFInvokingManager.OFBuildedListInvoker)}{evaluateEvent}}
 will be setted to null.
\begin{description}
\item[Parameter] ~
\begin{description}
\item[inputs]
ordered in accord with the ontological definition of the events trhougth the
 object property \verb!"hasTypeEventDefinition!
\end{description}
\item[Rückgabewert] 
true if the inputs are corrects. If return else, event computation dennied.
\end{description}
\item[{\ltdHypertarget{ontologyFramework.OFEventManagement.OFLogicalEventManagement.OFEventInterface.evaluateEvent(java.util.List<ontologyFramework.OFEventManagement.EventComputedData>,ontologyFramework.OFRunning.OFInvokingManager.OFBuildedListInvoker)}{evaluateEvent}\label{ontologyFramework.OFEventManagement.OFLogicalEventManagement.OFEventInterface.evaluateEvent(java.util.List<ontologyFramework.OFEventManagement.EventComputedData>,ontologyFramework.OFRunning.OFInvokingManager.OFBuildedListInvoker)}}]
~ implements how compute the event results starting from the inputs retrieved in
 \noprint{@link:OFEventParameterDefinition#getParameter()}\texttt{\hyperlink{ontologyFramework.OFEventManagement.OFEventParameterDefinition.getParameter()}{getParameter}}
\begin{description}
\item[Parameter] ~
\begin{description}
\item[inputs]
parameter
\item[invoker]
access to a builded class duriing software initialization
\end{description}
\item[Rückgabewert] 
true f the event occurs, false otherwise.
\end{description}
\end{description}
