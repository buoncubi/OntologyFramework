   % /--------------------------------------------\
   % | API-Dokumentation für einige Java-Packages |
   % |    (genaueres siehe doku-main.tex).        |
   % | LaTeX-Ausgabe erstellt von 'ltxdoclet'.    |
   % | Dieses Programm stammt von Paul Ebermann.  |
   % \--------------------------------------------/

   % Api-Dokumentation für Klasse ontologyFramework.OFEventManagement.OFTimeTriggerManagement.OFTimeTriggerImplementation.TriggFrequently (noch nicht fertig). 
\section[TriggFrequently]{Klasse \ltdHypertarget{ontologyFramework.OFEventManagement.OFTimeTriggerManagement.OFTimeTriggerImplementation.TriggFrequently-class}{ontologyFramework.OFEventManagement.OFTimeTriggerManagement.OFTimeTriggerImplementation.TriggFrequently}}\label{ontologyFramework.OFEventManagement.OFTimeTriggerManagement.OFTimeTriggerImplementation.TriggFrequently-class}
\subsection{Übersicht}
This class create a new Quartz Trigger with particular parameter. 
 The ontology must contain an individual which as those properties:
 {\ttfamily
\mbox{ }\mbox{}\newline
\mbox{ }	\verb!implementsOFTimeTriggerName "ontologyFramework.OFEventManagement.OFTimeTriggerManagement.OFTimeTriggerImplementation.TriggFrequently"^^string!\mbox{}\newline
\mbox{ }\mbox{ }\verb!&!\mbox{}\newline
\mbox{ }\mbox{ }\verb!in:ontologyFramework.OFEventManagement.OFEventParameter.
			?frequency 10.AsInteger
			?couter ^.AsInteger
			?priority 6.AsInteger
			!r triggFrequentlyInSeconds( ?frequency ?priority ?couter)"^^strign!\mbox{}\newline
\mbox{ }}
 
 So the method \noprint{@link:#isCorrectInput(List)}\texttt{\hyperlink{ontologyFramework.OFEventManagement.OFTimeTriggerManagement.OFTimeTriggerImplementation.TriggFrequently.isCorrectInput(java.util.List<ontologyFramework.OFEventManagement.EventComputedData>)}{isCorrectInput}} returns true only if: \verb!inputs.get(0).getParameter() instanceof Integer!, \verb!inputs.get(1).getParameter() instanceof Integer!
 and \verb!(inputs.get(2).getParameter() == null) || (inputs.get(2).getParameter() instanceof Integer)! are true.
 While the method \noprint{@link:#getTrigger(List, OFBuildedListInvoker)}\texttt{\hyperlink{ontologyFramework.OFEventManagement.OFTimeTriggerManagement.OFTimeTriggerImplementation.TriggFrequently.getTrigger(java.util.List<ontologyFramework.OFEventManagement.EventComputedData>,ontologyFramework.OFRunning.OFInvokingManager.OFBuildedListInvoker)}{getTrigger}} returns a quartz trigger with
 the specified parameter. If count number is equal to null than, the trigger has 
 \verb!repeatForever()! property.
 
 if counter is 0 than the trigger is \textquotedbl fired now\textquotedbl  only once
\begin{description}
\item[@author] 
Buoncomapgni Luca
\item[@version] 
1.0
\end{description}
\subsection{Inhaltsverzeichnis}
\subsection{Konstruktoren}
\begin{description}
\item[{\ltdHypertarget{ontologyFramework.OFEventManagement.OFTimeTriggerManagement.OFTimeTriggerImplementation.TriggFrequently()}{TriggFrequently}\label{ontologyFramework.OFEventManagement.OFTimeTriggerManagement.OFTimeTriggerImplementation.TriggFrequently()}}]
~ 
\end{description}
\subsection{Methoden}
\begin{description}
\item[{\ltdHypertarget{ontologyFramework.OFEventManagement.OFTimeTriggerManagement.OFTimeTriggerImplementation.TriggFrequently.isCorrectInput(java.util.List<ontologyFramework.OFEventManagement.EventComputedData>)}{isCorrectInput}\label{ontologyFramework.OFEventManagement.OFTimeTriggerManagement.OFTimeTriggerImplementation.TriggFrequently.isCorrectInput(java.util.List<ontologyFramework.OFEventManagement.EventComputedData>)}}]
~ 
\item[{\ltdHypertarget{ontologyFramework.OFEventManagement.OFTimeTriggerManagement.OFTimeTriggerImplementation.TriggFrequently.getTrigger(java.util.List<ontologyFramework.OFEventManagement.EventComputedData>,ontologyFramework.OFRunning.OFInvokingManager.OFBuildedListInvoker)}{getTrigger}\label{ontologyFramework.OFEventManagement.OFTimeTriggerManagement.OFTimeTriggerImplementation.TriggFrequently.getTrigger(java.util.List<ontologyFramework.OFEventManagement.EventComputedData>,ontologyFramework.OFRunning.OFInvokingManager.OFBuildedListInvoker)}}]
~ 
\end{description}
