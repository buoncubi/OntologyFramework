   % /--------------------------------------------\
   % | API-Dokumentation für einige Java-Packages |
   % |    (genaueres siehe doku-main.tex).        |
   % | LaTeX-Ausgabe erstellt von 'ltxdoclet'.    |
   % | Dieses Programm stammt von Paul Ebermann.  |
   % \--------------------------------------------/

   % Api-Dokumentation für Klasse ontologyFramework.OFEventManagement.OFTimeTriggerManagement.OFTimeTriggerInterface (noch nicht fertig). 
\section[OFTimeTriggerInterface]{Interface \ltdHypertarget{ontologyFramework.OFEventManagement.OFTimeTriggerManagement.OFTimeTriggerInterface-class}{ontologyFramework.OFEventManagement.OFTimeTriggerManagement.OFTimeTriggerInterface}}\label{ontologyFramework.OFEventManagement.OFTimeTriggerManagement.OFTimeTriggerInterface-class}
\subsection{Übersicht}
This class is used to initialise, store and compute Temporal Trigger.
 \noprint{@link:#isCorrectInput(List)}\texttt{\hyperlink{ontologyFramework.OFEventManagement.OFTimeTriggerManagement.OFTimeTriggerInterface.isCorrectInput(java.util.List<ontologyFramework.OFEventManagement.EventComputedData>)}{isCorrectInput}} is called frist and if it returns true
 than \noprint{@link:#getTrigger(List, OFBuildedListInvoker)}\texttt{\hyperlink{ontologyFramework.OFEventManagement.OFTimeTriggerManagement.OFTimeTriggerInterface.getTrigger(java.util.List<ontologyFramework.OFEventManagement.EventComputedData>,ontologyFramework.OFRunning.OFInvokingManager.OFBuildedListInvoker)}{getTrigger}}$\}$ is called with the
 same inputs. This is by default done from \noprint{@link:OFTimeTriggerDefinition#compute(OFBuildedListInvoker)}\texttt{\hyperlink{ontologyFramework.OFEventManagement.OFTimeTriggerManagement.OFTimeTriggerDefinition.compute(ontologyFramework.OFRunning.OFInvokingManager.OFBuildedListInvoker)}{compute}}
\begin{description}
\item[@author] 
Buoncomapgni Luca
\item[@version] 
1.0
\end{description}
\subsection{Inhaltsverzeichnis}
\subsection{Methoden}
\begin{description}
\item[{\ltdHypertarget{ontologyFramework.OFEventManagement.OFTimeTriggerManagement.OFTimeTriggerInterface.isCorrectInput(java.util.List<ontologyFramework.OFEventManagement.EventComputedData>)}{isCorrectInput}\label{ontologyFramework.OFEventManagement.OFTimeTriggerManagement.OFTimeTriggerInterface.isCorrectInput(java.util.List<ontologyFramework.OFEventManagement.EventComputedData>)}}]
~ sviluppated with safety pourposes it is called to check
 if the type of parameter in inputs are correct. If this return
 false the event result of \noprint{@link:#getTrigger(List, OFBuildedListInvoker)}\texttt{\hyperlink{ontologyFramework.OFEventManagement.OFTimeTriggerManagement.OFTimeTriggerInterface.getTrigger(java.util.List<ontologyFramework.OFEventManagement.EventComputedData>,ontologyFramework.OFRunning.OFInvokingManager.OFBuildedListInvoker)}{getTrigger}}
 will be setted to null.
\begin{description}
\item[Parameter] ~
\begin{description}
\item[inputs]
ordered in accord with the ontological definition of the events trhougth the
 object property \verb!"hasTypeTimeTriggereDefinition!
\end{description}
\item[Rückgabewert] 
true if the inputs are corrects. If return else, event computation dennied.
\end{description}
\item[{\ltdHypertarget{ontologyFramework.OFEventManagement.OFTimeTriggerManagement.OFTimeTriggerInterface.getTrigger(java.util.List<ontologyFramework.OFEventManagement.EventComputedData>,ontologyFramework.OFRunning.OFInvokingManager.OFBuildedListInvoker)}{getTrigger}\label{ontologyFramework.OFEventManagement.OFTimeTriggerManagement.OFTimeTriggerInterface.getTrigger(java.util.List<ontologyFramework.OFEventManagement.EventComputedData>,ontologyFramework.OFRunning.OFInvokingManager.OFBuildedListInvoker)}}]
~ implements how to get the temporal trigger starting from the inputs retrieved in
 \noprint{@link:OFEventParameterDefinition#getParameter()}\texttt{\hyperlink{ontologyFramework.OFEventManagement.OFEventParameterDefinition.getParameter()}{getParameter}}
\begin{description}
\item[Parameter] ~
\begin{description}
\item[inputs]
parameter
\item[invoker]
access to a builded class duriing software initialization
\end{description}
\item[Rückgabewert] 
true f the event occurs, false otherwise.
\end{description}
\end{description}
