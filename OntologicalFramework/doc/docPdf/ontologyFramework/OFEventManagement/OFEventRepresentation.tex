   % /--------------------------------------------\
   % | API-Dokumentation für einige Java-Packages |
   % |    (genaueres siehe doku-main.tex).        |
   % | LaTeX-Ausgabe erstellt von 'ltxdoclet'.    |
   % | Dieses Programm stammt von Paul Ebermann.  |
   % \--------------------------------------------/

   % Api-Dokumentation für Klasse ontologyFramework.OFEventManagement.OFEventRepresentation (noch nicht fertig). 
\section[OFEventRepresentation]{Klasse \ltdHypertarget{ontologyFramework.OFEventManagement.OFEventRepresentation-class}{ontologyFramework.OFEventManagement.OFEventRepresentation}}\label{ontologyFramework.OFEventManagement.OFEventRepresentation-class}
\subsection{Übersicht}
This class contains a basic implementation of how store initializate mechanism
 from the OFlanguage in data property.
 Thasnks to this class it is possible to use the Event mapping mecchanism 
 semply defining how to compute them.
\begin{description}
\item[@author] 
Buoncomapgni Luca
\item[@version] 
1.0
\end{description}
\subsection{Inhaltsverzeichnis}
\subsection{Konstruktoren}
\begin{description}
\item[{\ltdHypertarget{ontologyFramework.OFEventManagement.OFEventRepresentation(java.lang.String)}{OFEventRepresentation}\label{ontologyFramework.OFEventManagement.OFEventRepresentation(java.lang.String)}}]
~ Create a new definition of event
\begin{description}
\item[Parameter] ~
\begin{description}
\item[packageClassName]
the full qualify to a class that represent the event provedure 
 implementing \noprint{@link:OFEventInterface}\texttt{\hyperlink{ontologyFramework.OFEventManagement.OFLogicalEventManagement.OFEventInterface-class}{OFEventInterface}}
\end{description}
\end{description}
\end{description}
\subsection{Methoden}
\begin{description}
\item[{\ltdHypertarget{ontologyFramework.OFEventManagement.OFEventRepresentation.getOrder()}{getOrder}\label{ontologyFramework.OFEventManagement.OFEventRepresentation.getOrder()}}]
~ 
\begin{description}
\item[Rückgabewert] 
varNameOrder the ordered variable names to compute parameter for the event.
\end{description}
\item[{\ltdHypertarget{ontologyFramework.OFEventManagement.OFEventRepresentation.setOrder(java.util.List<java.lang.String>)}{setOrder}\label{ontologyFramework.OFEventManagement.OFEventRepresentation.setOrder(java.util.List<java.lang.String>)}}]
~ 
\begin{description}
\item[Parameter] ~
\begin{description}
\item[varNameOrder]
the ordered variable names to compute parameter for the event.
\end{description}
\end{description}
\item[{\ltdHypertarget{ontologyFramework.OFEventManagement.OFEventRepresentation.getClassName()}{getClassName}\label{ontologyFramework.OFEventManagement.OFEventRepresentation.getClassName()}}]
~ 
\begin{description}
\item[Rückgabewert] 
the fully java quilifier to the event class that implements \noprint{@link:OFEventInterface}\texttt{\hyperlink{ontologyFramework.OFEventManagement.OFLogicalEventManagement.OFEventInterface-class}{OFEventInterface}}
\end{description}
\item[{\ltdHypertarget{ontologyFramework.OFEventManagement.OFEventRepresentation.getParameterMap()}{getParameterMap}\label{ontologyFramework.OFEventManagement.OFEventRepresentation.getParameterMap()}}]
~ 
\begin{description}
\item[Rückgabewert] 
the parameterMap. It contains an unordered set of \noprint{@link:OFEventParameterDefinition}\texttt{\hyperlink{ontologyFramework.OFEventManagement.OFEventParameterDefinition-class}{OFEventParameterDefinition}} linked
 by variableName string value.
\end{description}
\item[{\ltdHypertarget{ontologyFramework.OFEventManagement.OFEventRepresentation.setParameterMap(java.util.Map<java.lang.String,ontologyFramework.OFEventManagement.OFEventParameterDefinition>)}{setParameterMap}\label{ontologyFramework.OFEventManagement.OFEventRepresentation.setParameterMap(java.util.Map<java.lang.String,ontologyFramework.OFEventManagement.OFEventParameterDefinition>)}}]
~ 
\begin{description}
\item[Parameter] ~
\begin{description}
\item[parameterMap]
set the parameterMap. It contains an unordered set of \noprint{@link:OFEventParameterDefinition}\texttt{\hyperlink{ontologyFramework.OFEventManagement.OFEventParameterDefinition-class}{OFEventParameterDefinition}} linked
 by variableName string value.
\end{description}
\end{description}
\item[{\ltdHypertarget{ontologyFramework.OFEventManagement.OFEventRepresentation.addToParameterMap(java.lang.String,ontologyFramework.OFEventManagement.OFEventParameterDefinition)}{addToParameterMap}\label{ontologyFramework.OFEventManagement.OFEventRepresentation.addToParameterMap(java.lang.String,ontologyFramework.OFEventManagement.OFEventParameterDefinition)}}]
~ add a parameter into the event tagged by its variable name. Those names must be 
 coherent with the one retrieved during the event building; managed
 by \noprint{@link:OFEventBuilder}\texttt{\hyperlink{ontologyFramework.OFEventManagement.OFLogicalEventManagement.OFEventBuilder-class}{OFEventBuilder}}
\begin{description}
\item[Parameter] ~
\begin{description}
\item[varName]
the name of the parameter
\item[epd]
a parameter to inject as input into the Event implementstion 
 (interface of \noprint{@link:OFEventInterface}\texttt{\hyperlink{ontologyFramework.OFEventManagement.OFLogicalEventManagement.OFEventInterface-class}{OFEventInterface}})
\end{description}
\end{description}
\item[{\ltdHypertarget{ontologyFramework.OFEventManagement.OFEventRepresentation.addToParameterMap(java.util.Map<java.lang.String,ontologyFramework.OFEventManagement.OFEventParameterDefinition>)}{addToParameterMap}\label{ontologyFramework.OFEventManagement.OFEventRepresentation.addToParameterMap(java.util.Map<java.lang.String,ontologyFramework.OFEventManagement.OFEventParameterDefinition>)}}]
~ add parameters into the event as a map where keys are variable names and
 value initialized parameter. The names, must be coherent with the one retrieved 
 during the event building managed by \noprint{@link:OFEventBuilder}\texttt{\hyperlink{ontologyFramework.OFEventManagement.OFLogicalEventManagement.OFEventBuilder-class}{OFEventBuilder}}
\begin{description}
\item[Parameter] ~
\begin{description}
\item[map]
of varName and parameter to inject as input into the Event implementation 
 (interface of \noprint{@link:OFEventInterface}\texttt{\hyperlink{ontologyFramework.OFEventManagement.OFLogicalEventManagement.OFEventInterface-class}{OFEventInterface}})
\end{description}
\end{description}
\item[{\ltdHypertarget{ontologyFramework.OFEventManagement.OFEventRepresentation.removeFromParameterMap(java.lang.String)}{removeFromParameterMap}\label{ontologyFramework.OFEventManagement.OFEventRepresentation.removeFromParameterMap(java.lang.String)}}]
~ 
\begin{description}
\item[Parameter] ~
\begin{description}
\item[varName]
name of the variable which define the parameter to remove
 from this event.
\end{description}
\end{description}
\item[{\ltdHypertarget{ontologyFramework.OFEventManagement.OFEventRepresentation.getComputedParameterList()}{getComputedParameterList}\label{ontologyFramework.OFEventManagement.OFEventRepresentation.getComputedParameterList()}}]
~ It goes trough all the parameter following the ordered name of variables.
 For each of them it instantiates an new \noprint{@link:EventComputedData}\texttt{\hyperlink{ontologyFramework.OFEventManagement.EventComputedData-class}{EventComputedData}} with the 
 correspondent parameter (computed each time using \noprint{@link:OFEventParameterDefinition#getParameter()}\texttt{\hyperlink{ontologyFramework.OFEventManagement.OFEventParameterDefinition.getParameter()}{getParameter}})
 and its ontology reference. All of them are than collected in a ordered List.
\begin{description}
\item[Rückgabewert] 
update computed list of parameter results
\end{description}
\item[{\ltdHypertarget{ontologyFramework.OFEventManagement.OFEventRepresentation.compute(ontologyFramework.OFRunning.OFInvokingManager.OFBuildedListInvoker)}{compute}\label{ontologyFramework.OFEventManagement.OFEventRepresentation.compute(ontologyFramework.OFRunning.OFInvokingManager.OFBuildedListInvoker)}}]
~ $/$**
 Here the creation of a new instance of the event implementation should be done
 ( by default for \noprint{@link:OFTimeTriggerInterface}\texttt{\hyperlink{ontologyFramework.OFEventManagement.OFTimeTriggerManagement.OFTimeTriggerInterface-class}{OFTimeTriggerInterface}} and \noprint{@link:OFEventInterface}\texttt{\hyperlink{ontologyFramework.OFEventManagement.OFLogicalEventManagement.OFEventInterface-class}{OFEventInterface}}. 
 Using \noprint{@link:#getClassName()}\texttt{\hyperlink{ontologyFramework.OFEventManagement.OFEventRepresentation.getClassName()}{getClassName}} is possible to load a new instance of such
 Interface (you must define your own methods to do so). Than the ordered and 
 update parameter values can be retrieved using \noprint{@link:OFEventRepresentation#getComputedParameterList()}\texttt{\hyperlink{ontologyFramework.OFEventManagement.OFEventRepresentation.getComputedParameterList()}{getComputedParameterList}},
 to have the inputs to check your own event implementation.
\begin{description}
\item[Parameter] ~
\begin{description}
\item[invoker]
lsit of builded class during initialization to be used by \noprint{@link:OFEventInterface#evaluateEvent(List, OFBuildedListInvoker)}\texttt{\hyperlink{ontologyFramework.OFEventManagement.OFLogicalEventManagement.OFEventInterface.evaluateEvent(java.util.List<ontologyFramework.OFEventManagement.EventComputedData>,ontologyFramework.OFRunning.OFInvokingManager.OFBuildedListInvoker)}{evaluateEvent}}
\end{description}
\item[Rückgabewert] 
the event result
\end{description}
\end{description}
