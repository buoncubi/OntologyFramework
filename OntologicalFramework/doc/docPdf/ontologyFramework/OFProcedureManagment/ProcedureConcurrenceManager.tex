   % /--------------------------------------------\
   % | API-Dokumentation für einige Java-Packages |
   % |    (genaueres siehe doku-main.tex).        |
   % | LaTeX-Ausgabe erstellt von 'ltxdoclet'.    |
   % | Dieses Programm stammt von Paul Ebermann.  |
   % \--------------------------------------------/

   % Api-Dokumentation für Klasse ontologyFramework.OFProcedureManagment.ProcedureConcurrenceManager (noch nicht fertig). 
\section[ProcedureConcurrenceManager]{Klasse \ltdHypertarget{ontologyFramework.OFProcedureManagment.ProcedureConcurrenceManager-class}{ontologyFramework.OFProcedureManagment.ProcedureConcurrenceManager}}\label{ontologyFramework.OFProcedureManagment.ProcedureConcurrenceManager-class}
\subsection{Übersicht}
This class is used to initialize and manage a Procedure with its ID which
 is necessary to manage a concurrent pool approach.
 Practically, this class represent a List of \noprint{@link:ProcedureConcurrenceData}\texttt{\hyperlink{ontologyFramework.OFProcedureManagment.ProcedureConcurrenceData-class}{ProcedureConcurrenceData}}
 with fixed size.
 Initially all places of the list are null and this means that the pool
 is empty. Where an instance is running, the first place of the list will
 be associate to a ProcedureConcurrenceData object and it will means
 that that place inside the pool is unusable. Than, when the instance
 Finishes its work its relate place inside the list will go back to null.
 So, if this list does not contain null places means that the pool
 is full, namely the procedure cannot run even if all its conditions
 are satisfied. The ID associate to the procedure will be equal to the index 
 inside the list represented by this class.
\begin{description}
\item[@author] 
Buoncomapgni Luca
\item[@version] 
1.0
\end{description}
\subsection{Inhaltsverzeichnis}
\subsection{Konstruktoren}
\begin{description}
\item[{\ltdHypertarget{ontologyFramework.OFProcedureManagment.ProcedureConcurrenceManager(java.lang.String,java.lang.Integer)}{ProcedureConcurrenceManager}\label{ontologyFramework.OFProcedureManagment.ProcedureConcurrenceManager(java.lang.String,java.lang.Integer)}}]
~ Create a new ProcedureConcurrenceManager associate to a particular 
 Procedure individual.
\begin{description}
\item[Parameter] ~
\begin{description}
\item[procedureName]
ontological name of the procedure
\item[concurrencyOrder]
the concurrency pool size for this procedure.
\end{description}
\end{description}
\end{description}
\subsection{Methoden}
\begin{description}
\item[{\ltdHypertarget{ontologyFramework.OFProcedureManagment.ProcedureConcurrenceManager.generateID()}{generateID}\label{ontologyFramework.OFProcedureManagment.ProcedureConcurrenceManager.generateID()}}]
~ Returns an new instance of \noprint{@link:ProcedureConcurrenceData}\texttt{\hyperlink{ontologyFramework.OFProcedureManagment.ProcedureConcurrenceData-class}{ProcedureConcurrenceData}}
 initialized with procedureName and the index (ID)
 inside the concurrent pool. If the pool is full than this 
 method will return null.
\begin{description}
\item[Rückgabewert] 
the procedure data in terms of concurrency if the thread pool is
 not full.
\end{description}
\item[{\ltdHypertarget{ontologyFramework.OFProcedureManagment.ProcedureConcurrenceManager.removeID(ontologyFramework.OFProcedureManagment.ProcedureConcurrenceData)}{removeID}\label{ontologyFramework.OFProcedureManagment.ProcedureConcurrenceManager.removeID(ontologyFramework.OFProcedureManagment.ProcedureConcurrenceData)}}]
~ remove a procedure from the concurrency pool.
 It should be called as soon as the procedure ends.
\begin{description}
\item[Parameter] ~
\begin{description}
\item[pcd]
concurrency data with respect to a procedure ended that
 should be removed from the thread pool.
\end{description}
\end{description}
\item[{\ltdHypertarget{ontologyFramework.OFProcedureManagment.ProcedureConcurrenceManager.getProcedureName()}{getProcedureName}\label{ontologyFramework.OFProcedureManagment.ProcedureConcurrenceManager.getProcedureName()}}]
~ 
\begin{description}
\item[Rückgabewert] 
the procedureName
\end{description}
\item[{\ltdHypertarget{ontologyFramework.OFProcedureManagment.ProcedureConcurrenceManager.getProcedureActivated()}{getProcedureActivated}\label{ontologyFramework.OFProcedureManagment.ProcedureConcurrenceManager.getProcedureActivated()}}]
~ 
\begin{description}
\item[Rückgabewert] 
the list of all the instances running inside the pool related to 
 this procedure.
\end{description}
\end{description}
