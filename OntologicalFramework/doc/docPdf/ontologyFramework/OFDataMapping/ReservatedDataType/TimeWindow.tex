   % /--------------------------------------------\
   % | API-Dokumentation für einige Java-Packages |
   % |    (genaueres siehe doku-main.tex).        |
   % | LaTeX-Ausgabe erstellt von 'ltxdoclet'.    |
   % | Dieses Programm stammt von Paul Ebermann.  |
   % \--------------------------------------------/

   % Api-Dokumentation für Klasse ontologyFramework.OFDataMapping.ReservatedDataType.TimeWindow (noch nicht fertig). 
\section[TimeWindow]{Klasse \ltdHypertarget{ontologyFramework.OFDataMapping.ReservatedDataType.TimeWindow-class}{ontologyFramework.OFDataMapping.ReservatedDataType.TimeWindow}}\label{ontologyFramework.OFDataMapping.ReservatedDataType.TimeWindow-class}
\subsection{Übersicht}
This class is the mapping representation of an ontological time windows.
  
 A time windows is represented in the ontology with an individual and a bunch
 of SWRL rules more addressed in \noprint{@link:ontologyFramework.OFDataMapping.complexDataType.TimeWindowsDataMapper}\texttt{\hyperlink{ontologyFramework.OFDataMapping.complexDataType.TimeWindowsDataMapper-class}{TimeWindowsDataMapper}}.
\begin{description}
\item[@author] 
Buoncomapgni Luca
\item[@version] 
1.0
\end{description}
\subsection{Inhaltsverzeichnis}
\subsection{Konstruktoren}
\begin{description}
\item[{\ltdHypertarget{ontologyFramework.OFDataMapping.ReservatedDataType.TimeWindow(java.lang.Long,java.lang.Long)}{TimeWindow}\label{ontologyFramework.OFDataMapping.ReservatedDataType.TimeWindow(java.lang.Long,java.lang.Long)}}]
~ Create a time window whic has a size and a center value with respect to the
 centre ( = 0 ) of an abstract time line always time invariant.
 For example give time windows as: <<@literal:T1( 10, 0), T2( 10, -15)>> 
 <<@literal:and T3( 10, +15)>> the representation will be distributed uniformally
 in an always fixed time line; as: 
 <<@literal:T2€[ -15, -5) & T1€[ -5, 5) & T3[5, 15)>>.
 During the running of the system this line will move during time
 and so the actual windows would be:
 <<@literal:T2€[ -15+t, -5+t) & T1€[ -5+t, 5+t) & T3[5+t, 15+t)>>
 Where <<@literal:t>> is a value close to the real time instance.
\begin{description}
\item[Parameter] ~
\begin{description}
\item[size]
number of millisecond of the windows size
\item[relativeCentre]
relative number of millisecond in which the windows is
 centered with respect to now.
\end{description}
\end{description}
\item[{\ltdHypertarget{ontologyFramework.OFDataMapping.ReservatedDataType.TimeWindow(java.lang.Long,java.lang.Long,java.lang.String,java.lang.String)}{TimeWindow}\label{ontologyFramework.OFDataMapping.ReservatedDataType.TimeWindow(java.lang.Long,java.lang.Long,java.lang.String,java.lang.String)}}]
~ It creates a time windows using the parameters: \verb!size! and 
 \verb!relative center! as sow in \noprint{@link:#TimeWindow(Long, Long)}\texttt{\hyperlink{ontologyFramework.OFDataMapping.ReservatedDataType.TimeWindow(java.lang.Long,java.lang.Long)}{TimeWindow}}.
 Moreover, it assign to this class names for ontological
 entities that are needed to map the windows from this framework to 
 the ontology. In particolar they are the name of an Individual belong
 to the ontological class <<@literal:DataType -> TimeWindow>>. And the name
 of the class in which other individual can be classified as belong to
 a give time window. <<@literal:DataType -> TimeRepresentation>> is
 the ontological path by default
\begin{description}
\item[Parameter] ~
\begin{description}
\item[size]
number of millisecond of the windows size
\item[relativeCentre]
relative number of millisecond in which the windows is
 centered with respect to now.
\item[individualName]
name of the individual that describe this time window in the 
 ontology
\item[className]
name of the class that will behave as
 a time windows from the reasoning point of view. This class will
 collect all the other individual of the ontology which has a time stamp
 property that fall on this windows of time.
\end{description}
\end{description}
\end{description}
\subsection{Methoden}
\begin{description}
\item[{\ltdHypertarget{ontologyFramework.OFDataMapping.ReservatedDataType.TimeWindow.getSize()}{getSize}\label{ontologyFramework.OFDataMapping.ReservatedDataType.TimeWindow.getSize()}}]
~ 
\begin{description}
\item[Rückgabewert] 
the size
\end{description}
\item[{\ltdHypertarget{ontologyFramework.OFDataMapping.ReservatedDataType.TimeWindow.getRelativeCentre()}{getRelativeCentre}\label{ontologyFramework.OFDataMapping.ReservatedDataType.TimeWindow.getRelativeCentre()}}]
~ 
\begin{description}
\item[Rückgabewert] 
the relativeCentre
\end{description}
\item[{\ltdHypertarget{ontologyFramework.OFDataMapping.ReservatedDataType.TimeWindow.getIndividualName()}{getIndividualName}\label{ontologyFramework.OFDataMapping.ReservatedDataType.TimeWindow.getIndividualName()}}]
~ 
\begin{description}
\item[Rückgabewert] 
the individual
\end{description}
\item[{\ltdHypertarget{ontologyFramework.OFDataMapping.ReservatedDataType.TimeWindow.setIndividualName(java.lang.String)}{setIndividualName}\label{ontologyFramework.OFDataMapping.ReservatedDataType.TimeWindow.setIndividualName(java.lang.String)}}]
~ 
\begin{description}
\item[Parameter] ~
\begin{description}
\item[individualName]

\end{description}
\end{description}
\item[{\ltdHypertarget{ontologyFramework.OFDataMapping.ReservatedDataType.TimeWindow.getClassName()}{getClassName}\label{ontologyFramework.OFDataMapping.ReservatedDataType.TimeWindow.getClassName()}}]
~ 
\begin{description}
\item[Rückgabewert] 
the className
\end{description}
\item[{\ltdHypertarget{ontologyFramework.OFDataMapping.ReservatedDataType.TimeWindow.setClassName(java.lang.String)}{setClassName}\label{ontologyFramework.OFDataMapping.ReservatedDataType.TimeWindow.setClassName(java.lang.String)}}]
~ 
\begin{description}
\item[Parameter] ~
\begin{description}
\item[className]
the className to set
\end{description}
\end{description}
\item[{\ltdHypertarget{ontologyFramework.OFDataMapping.ReservatedDataType.TimeWindow.getRootClass()}{getRootClass}\label{ontologyFramework.OFDataMapping.ReservatedDataType.TimeWindow.getRootClass()}}]
~ 
\begin{description}
\item[Rückgabewert] 
the rootClass
\end{description}
\item[{\ltdHypertarget{ontologyFramework.OFDataMapping.ReservatedDataType.TimeWindow.setRootClass(java.lang.String)}{setRootClass}\label{ontologyFramework.OFDataMapping.ReservatedDataType.TimeWindow.setRootClass(java.lang.String)}}]
~ 
\begin{description}
\item[Parameter] ~
\begin{description}
\item[rootClass]
the rootClass to set
\end{description}
\end{description}
\item[{\ltdHypertarget{ontologyFramework.OFDataMapping.ReservatedDataType.TimeWindow.toString()}{toString}\label{ontologyFramework.OFDataMapping.ReservatedDataType.TimeWindow.toString()}}]
~ 
\item[{\ltdHypertarget{ontologyFramework.OFDataMapping.ReservatedDataType.TimeWindow.getKeyWord()}{getKeyWord}\label{ontologyFramework.OFDataMapping.ReservatedDataType.TimeWindow.getKeyWord()}}]
~ 
\begin{description}
\item[Rückgabewert] 
the keyWord
\end{description}
\item[{\ltdHypertarget{ontologyFramework.OFDataMapping.ReservatedDataType.TimeWindow.setKeyWord(java.lang.String[])}{setKeyWord}\label{ontologyFramework.OFDataMapping.ReservatedDataType.TimeWindow.setKeyWord(java.lang.String[])}}]
~ 
\begin{description}
\item[Parameter] ~
\begin{description}
\item[keyWord]
the keyWord to set
\end{description}
\end{description}
\item[{\ltdHypertarget{ontologyFramework.OFDataMapping.ReservatedDataType.TimeWindow.getAbsoluteTimeWindows(java.lang.Long)}{getAbsoluteTimeWindows}\label{ontologyFramework.OFDataMapping.ReservatedDataType.TimeWindow.getAbsoluteTimeWindows(java.lang.Long)}}]
~ return the windows with its size and central instant computed
 with respect to an reference clock value. (long unix time stamp in milliseconds)
\begin{description}
\item[Parameter] ~
\begin{description}
\item[actualCk]
time stamp of when compute the windows
\end{description}
\item[Rückgabewert] 
time windows compute with respect to an actual referiment.
\end{description}
\item[{\ltdHypertarget{ontologyFramework.OFDataMapping.ReservatedDataType.TimeWindow.getAbsoluteTimeWindows()}{getAbsoluteTimeWindows}\label{ontologyFramework.OFDataMapping.ReservatedDataType.TimeWindow.getAbsoluteTimeWindows()}}]
~ 
\begin{description}
\item[Rückgabewert] 
the absulute time windows computed for the time which this method is called
 
 Return the result of: \noprint{@link:#getAbsoluteTimeWindows(Long)}\texttt{\hyperlink{ontologyFramework.OFDataMapping.ReservatedDataType.TimeWindow.getAbsoluteTimeWindows(java.lang.Long)}{getAbsoluteTimeWindows}} \verb!(System.currentTimeMillis())!
\end{description}
\item[{\ltdHypertarget{ontologyFramework.OFDataMapping.ReservatedDataType.TimeWindow.getAbsoluteTimeWindows(ontologyFramework.OFContextManagement.OWLReferences)}{getAbsoluteTimeWindows}\label{ontologyFramework.OFDataMapping.ReservatedDataType.TimeWindow.getAbsoluteTimeWindows(ontologyFramework.OFContextManagement.OWLReferences)}}]
~ return the windows with its size and central instant  as are descripted 
 in the ontology refered from \verb!ontoRef!. It must contain the data
 property \verb!keyWord[ 4] = "hasTypeTimeWindowsUpperBound! 
 and \verb!keyWord[ 5] = "hasTypeTimeWindowsLowerBound!.
\begin{description}
\item[Parameter] ~
\begin{description}
\item[ontoRef]
time stamp of when compute the windows
\end{description}
\item[Rückgabewert] 
time windows compute with respect to an actual referiment.
\end{description}
\end{description}
