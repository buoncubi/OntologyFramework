   % /--------------------------------------------\
   % | API-Dokumentation für einige Java-Packages |
   % |    (genaueres siehe doku-main.tex).        |
   % | LaTeX-Ausgabe erstellt von 'ltxdoclet'.    |
   % | Dieses Programm stammt von Paul Ebermann.  |
   % \--------------------------------------------/

   % Api-Dokumentation für Klasse ontologyFramework.OFDataMapping.ReservatedDataType.AbsoluteTimeWindow (noch nicht fertig). 
\section[AbsoluteTimeWindow]{Klasse \ltdHypertarget{ontologyFramework.OFDataMapping.ReservatedDataType.AbsoluteTimeWindow-class}{ontologyFramework.OFDataMapping.ReservatedDataType.AbsoluteTimeWindow}}\label{ontologyFramework.OFDataMapping.ReservatedDataType.AbsoluteTimeWindow-class}
\subsection{Übersicht}
This class defines the Object which implements an AbsoluteTimeWindows.
 It should be use to refer to a time windows which has its place
 in a time line. Basically it is just a data structure to store
 the state of a time windows in a particular instant.
 Note that in this framework time instances are describe has a Long 
 which represents a Unix time stamp.
\begin{description}
\item[@author] 
Buoncomapgni Luca
\item[@version] 
1.0
\end{description}
\subsection{Inhaltsverzeichnis}
\subsection{Konstruktoren}
\begin{description}
\item[{\ltdHypertarget{ontologyFramework.OFDataMapping.ReservatedDataType.AbsoluteTimeWindow(java.lang.Long,java.lang.Long,java.lang.Long,java.lang.Long)}{AbsoluteTimeWindow}\label{ontologyFramework.OFDataMapping.ReservatedDataType.AbsoluteTimeWindow(java.lang.Long,java.lang.Long,java.lang.Long,java.lang.Long)}}]
~ Create absolute time windows with final property set.
\begin{description}
\item[Parameter] ~
\begin{description}
\item[lowerBound]
minimum time stamp of the windows
\item[centralTime]
central time stamp of the windows
\item[upperBound]
maximum time stamp of the windows
\item[ck]
time instant fixed in the representation. It should be the time
 when this windows has been frozen.
\end{description}
\end{description}
\end{description}
\subsection{Methoden}
\begin{description}
\item[{\ltdHypertarget{ontologyFramework.OFDataMapping.ReservatedDataType.AbsoluteTimeWindow.getUpperBound()}{getUpperBound}\label{ontologyFramework.OFDataMapping.ReservatedDataType.AbsoluteTimeWindow.getUpperBound()}}]
~ 
\begin{description}
\item[Rückgabewert] 
the upperBound
\end{description}
\item[{\ltdHypertarget{ontologyFramework.OFDataMapping.ReservatedDataType.AbsoluteTimeWindow.getLowerBound()}{getLowerBound}\label{ontologyFramework.OFDataMapping.ReservatedDataType.AbsoluteTimeWindow.getLowerBound()}}]
~ 
\begin{description}
\item[Rückgabewert] 
the lowerBound
\end{description}
\item[{\ltdHypertarget{ontologyFramework.OFDataMapping.ReservatedDataType.AbsoluteTimeWindow.getCentralTime()}{getCentralTime}\label{ontologyFramework.OFDataMapping.ReservatedDataType.AbsoluteTimeWindow.getCentralTime()}}]
~ 
\begin{description}
\item[Rückgabewert] 
the centralTime
\end{description}
\item[{\ltdHypertarget{ontologyFramework.OFDataMapping.ReservatedDataType.AbsoluteTimeWindow.getActualClock()}{getActualClock}\label{ontologyFramework.OFDataMapping.ReservatedDataType.AbsoluteTimeWindow.getActualClock()}}]
~ 
\begin{description}
\item[Rückgabewert] 
the actualClock when the framework decide to froze the windows in this class
\end{description}
\end{description}
