   % /--------------------------------------------\
   % | API-Dokumentation für einige Java-Packages |
   % |    (genaueres siehe doku-main.tex).        |
   % | LaTeX-Ausgabe erstellt von 'ltxdoclet'.    |
   % | Dieses Programm stammt von Paul Ebermann.  |
   % \--------------------------------------------/

   % Api-Dokumentation für Klasse ontologyFramework.OFDataMapping.ReservatedDataType.KeyWordsMapper (noch nicht fertig). 
\section[KeyWordsMapper]{Klasse \ltdHypertarget{ontologyFramework.OFDataMapping.ReservatedDataType.KeyWordsMapper-class}{ontologyFramework.OFDataMapping.ReservatedDataType.KeyWordsMapper}}\label{ontologyFramework.OFDataMapping.ReservatedDataType.KeyWordsMapper-class}
\subsection{Übersicht}
This static class is used to represent a Data Type mapper
 which is by default defined in the framework.
 
 In particular it is able to map an Array of String as:\mbox{}\newline

 \verb!strings = ["A" "B" "C" ...]!
 w.r.t an ontological individual I which as the default DataProperty:\mbox{}\newline

 \verb!I {@value #propName} "A B C ..."^^string! 
 
 If key words exist in an individual that are builded from the 
 framework they are mapped with the porpuses to inject names
 in the builded classes. Basically to make thir coode more
 general with respect to different ontologies. 
 
 Note that this mapper does not store arrays in the ontology
 but only read them. In fact, due non trade-safe capability
 of the ontology, an ArrayMapper should be used for this kind
 of data types. Anyway this class permits the implementation
 of customizable approaches to describe the succession of data inside 
 an array, without affect the initialization procedure
 of the framework.
\begin{description}
\item[@author] 
Buoncomapgni Luca
\item[@version] 
1.0
\end{description}
\subsection{Inhaltsverzeichnis}
\subsection{Variablen}
\begin{description}
\item[{\ltdHypertarget{ontologyFramework.OFDataMapping.ReservatedDataType.KeyWordsMapper.KEYWORD_propName}{KEYWORD\_propName}\label{ontologyFramework.OFDataMapping.ReservatedDataType.KeyWordsMapper.KEYWORD_propName}}]
~ Defines the default name of the ontologica Data Property
 to map KeyWords between the system and the data structure.
\end{description}
\subsection{Methoden}
\begin{description}
\item[{\ltdHypertarget{ontologyFramework.OFDataMapping.ReservatedDataType.KeyWordsMapper.getKeyWordFromOntology(java.lang.String,ontologyFramework.OFContextManagement.OWLReferences)}{getKeyWordFromOntology}\label{ontologyFramework.OFDataMapping.ReservatedDataType.KeyWordsMapper.getKeyWordFromOntology(java.lang.String,ontologyFramework.OFContextManagement.OWLReferences)}}]
~ Given the name of an individual it retrieve that specific
 individual. Than, it calls \noprint{@link:#getKeyWordFromOntology(OWLNamedIndividual, OWLReferences)}\texttt{\hyperlink{ontologyFramework.OFDataMapping.ReservatedDataType.KeyWordsMapper.getKeyWordFromOntology(org.semanticweb.owlapi.model.OWLNamedIndividual,ontologyFramework.OFContextManagement.OWLReferences)}{getKeyWordFromOntology}}
 and the returning value is propagated.
\begin{description}
\item[Parameter] ~
\begin{description}
\item[individualName]
name of the ontological individual for which get the key words.
\item[ontoRef]
OWL references to the ontology.
\end{description}
\item[Rückgabewert] 
an Array of string where every cell contains a key word
\end{description}
\item[{\ltdHypertarget{ontologyFramework.OFDataMapping.ReservatedDataType.KeyWordsMapper.getKeyWordFromOntology(org.semanticweb.owlapi.model.OWLNamedIndividual,ontologyFramework.OFContextManagement.OWLReferences)}{getKeyWordFromOntology}\label{ontologyFramework.OFDataMapping.ReservatedDataType.KeyWordsMapper.getKeyWordFromOntology(org.semanticweb.owlapi.model.OWLNamedIndividual,ontologyFramework.OFContextManagement.OWLReferences)}}]
~ Given an ontological individual which has a data property <<@value:#KEYWORD_propName>>
 it returns an array of string containing, all the word (in each
 cell a word separated by \textquotedbl  \textquotedbl ). The result can be different if the property
 does not contain string ( see \noprint{@link:OWLLiteral#getLiteral()}\texttt{getLiteral} for more).
 It returns \verb!Null! if the individual does not exist or if
 it does not have such data property.
\begin{description}
\item[Parameter] ~
\begin{description}
\item[individual]
ontological individual from where retrieve the key words
\item[ontoRef]
OWL references to the ontology
\end{description}
\item[Rückgabewert] 
the key words collected in an array of strings.
\end{description}
\end{description}
