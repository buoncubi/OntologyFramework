   % /--------------------------------------------\
   % | API-Dokumentation für einige Java-Packages |
   % |    (genaueres siehe doku-main.tex).        |
   % | LaTeX-Ausgabe erstellt von 'ltxdoclet'.    |
   % | Dieses Programm stammt von Paul Ebermann.  |
   % \--------------------------------------------/

   % Api-Dokumentation für Klasse ontologyFramework.OFContextManagement.OWLLibrary (noch nicht fertig). 
\section[OWLLibrary]{Klasse \ltdHypertarget{ontologyFramework.OFContextManagement.OWLLibrary-class}{ontologyFramework.OFContextManagement.OWLLibrary}}\label{ontologyFramework.OFContextManagement.OWLLibrary-class}
\subsection{Übersicht}
This static class implement several common procedure
 for manipulating entity inside an ontology, using
 OWL api 3.0
\begin{description}
\item[@author] 
Buoncomapgni Luca
\item[@version] 
1.0
\end{description}
\subsection{Inhaltsverzeichnis}
\subsection{Variablen}
\begin{description}
\item[{\ltdHypertarget{ontologyFramework.OFContextManagement.OWLLibrary.PELLET_reasonerFactoryQualifier}{PELLET\_reasonerFactoryQualifier}\label{ontologyFramework.OFContextManagement.OWLLibrary.PELLET_reasonerFactoryQualifier}}]
~ Full qualifier of the Pellet reasoner Factory. String to be called by
 Java reflection to instantiate a Reasoner.
\item[{\ltdHypertarget{ontologyFramework.OFContextManagement.OWLLibrary.SNOROCKET_reasonerFactoryQualifier}{SNOROCKET\_reasonerFactoryQualifier}\label{ontologyFramework.OFContextManagement.OWLLibrary.SNOROCKET_reasonerFactoryQualifier}}]
~ Full qualifier of the Snorocket reasoner Factory. String to be called by
 Java reflection to instantiate a Reasoner.
\item[{\ltdHypertarget{ontologyFramework.OFContextManagement.OWLLibrary.HERMIT_reasonerFactoryQualifier}{HERMIT\_reasonerFactoryQualifier}\label{ontologyFramework.OFContextManagement.OWLLibrary.HERMIT_reasonerFactoryQualifier}}]
~ Full qualifier of the Hermit reasoner Factory. String to be called by
 Java reflection to instantiate a Reasoner.
\item[{\ltdHypertarget{ontologyFramework.OFContextManagement.OWLLibrary.FACTPLUSPLUS_reasonerFactoryQualifier}{FACTPLUSPLUS\_reasonerFactoryQualifier}\label{ontologyFramework.OFContextManagement.OWLLibrary.FACTPLUSPLUS_reasonerFactoryQualifier}}]
~ Full qualifier of the Fact++ reasoner Factory. String to be called by
 Java reflection to instantiate a Reasoner.
\end{description}
\subsection{Methoden}
\begin{description}
\item[{\ltdHypertarget{ontologyFramework.OFContextManagement.OWLLibrary.createOntologyManager(ontologyFramework.OFContextManagement.OWLReferences)}{createOntologyManager}\label{ontologyFramework.OFContextManagement.OWLLibrary.createOntologyManager(ontologyFramework.OFContextManagement.OWLReferences)}}]
~ creates and returns a new OWLOntologyManager. 
 If the parameter has a not null \noprint{@link:OWLReferences#getIriFilePath()}\texttt{\hyperlink{ontologyFramework.OFContextManagement.OWLReferences.getIriFilePath()}{getIriFilePath}} and
 \noprint{@link:OWLReferences#getOntologyPath()}\texttt{\hyperlink{ontologyFramework.OFContextManagement.OWLReferences.getOntologyPath()}{getOntologyPath}} that this method set this Iri mapper
 to the manager using: \verb!manager.addIRIMapper( new SimpleIRIMapper( ontoPath, filePath))!
\begin{description}
\item[Parameter] ~
\begin{description}
\item[OWLReferences]
reference to the ontology.
\end{description}
\item[Rückgabewert] 
the manager of the ontology refereed by the parameter.
\end{description}
\item[{\ltdHypertarget{ontologyFramework.OFContextManagement.OWLLibrary.createOntology(ontologyFramework.OFContextManagement.OWLReferences)}{createOntology}\label{ontologyFramework.OFContextManagement.OWLLibrary.createOntology(ontologyFramework.OFContextManagement.OWLReferences)}}]
~ Creates an new empty ontology in accord to the 
 \noprint{@link:OWLReferences#getIriOntologyPath()}\texttt{\hyperlink{ontologyFramework.OFContextManagement.OWLReferences.getIriOntologyPath()}{getIriOntologyPath}}. It will return null 
 if the ontology path associate to the parameter is null.
\begin{description}
\item[Parameter] ~
\begin{description}
\item[OWLReferences]
reference to the ontology.
\end{description}
\item[Exceptions] ~
\begin{description}
\item[OWLOntologyCreationException]

\end{description}
\item[Rückgabewert] 
a new empty ontology in accord with the parameter.
\end{description}
\item[{\ltdHypertarget{ontologyFramework.OFContextManagement.OWLLibrary.loadOntologyFromFile(ontologyFramework.OFContextManagement.OWLReferences)}{loadOntologyFromFile}\label{ontologyFramework.OFContextManagement.OWLLibrary.loadOntologyFromFile(ontologyFramework.OFContextManagement.OWLReferences)}}]
~ It loads an ontology from file in accord with the function parameter; 
 to do so the method uses the ontology manager from: \noprint{@link:OWLReferences#getManager()}\texttt{\hyperlink{ontologyFramework.OFContextManagement.OWLReferences.getManager()}{getManager}}.
 It will return null if \noprint{@link:OWLReferences#getIriOntologyPath()}\texttt{\hyperlink{ontologyFramework.OFContextManagement.OWLReferences.getIriOntologyPath()}{getIriOntologyPath}} is null.
\begin{description}
\item[Parameter] ~
\begin{description}
\item[OWLReferences]
reference to the ontology.
\end{description}
\item[Exceptions] ~
\begin{description}
\item[OWLOntologyCreationException]

\end{description}
\item[Rückgabewert] 
a pointer to the ontology refered by the parameter,
\end{description}
\item[{\ltdHypertarget{ontologyFramework.OFContextManagement.OWLLibrary.loadOntologyFromWeb(ontologyFramework.OFContextManagement.OWLReferences)}{loadOntologyFromWeb}\label{ontologyFramework.OFContextManagement.OWLLibrary.loadOntologyFromWeb(ontologyFramework.OFContextManagement.OWLReferences)}}]
~ It loads an ontology where its \noprint{@link:OWLReferences#getIriOntologyPath()}\texttt{\hyperlink{ontologyFramework.OFContextManagement.OWLReferences.getIriOntologyPath()}{getIriOntologyPath}} 
 defines an path to be browsed into the web. It returns null if the IRI
 ontology Path is null.
\begin{description}
\item[Parameter] ~
\begin{description}
\item[ontoRef]
reference to the ontology.
\end{description}
\item[Exceptions] ~
\begin{description}
\item[OWLOntologyCreationException]

\end{description}
\item[Rückgabewert] 
a pointer to the ontology refered by the parameter,
\end{description}
\item[{\ltdHypertarget{ontologyFramework.OFContextManagement.OWLLibrary.getPrefixFormat(ontologyFramework.OFContextManagement.OWLReferences)}{getPrefixFormat}\label{ontologyFramework.OFContextManagement.OWLLibrary.getPrefixFormat(ontologyFramework.OFContextManagement.OWLReferences)}}]
~ Returns a prefix manager to be attached into an ontolofy manager
 to simplify IRI definition and usage
\begin{description}
\item[Parameter] ~
\begin{description}
\item[OWLReferences]
a reference to an OWL ontology.
\end{description}
\item[Rückgabewert] 
a prefix manager format.
\end{description}
\item[{\ltdHypertarget{ontologyFramework.OFContextManagement.OWLLibrary.getOWLDataFactory(ontologyFramework.OFContextManagement.OWLReferences)}{getOWLDataFactory}\label{ontologyFramework.OFContextManagement.OWLLibrary.getOWLDataFactory(ontologyFramework.OFContextManagement.OWLReferences)}}]
~ Returns the OWLDataFactory associate to the OWLManager 
 associate to the parameter.
\begin{description}
\item[Parameter] ~
\begin{description}
\item[OWLReferences]

\end{description}
\item[Rückgabewert] 
an OWL data factory
\end{description}
\item[{\ltdHypertarget{ontologyFramework.OFContextManagement.OWLLibrary.getReasoner(java.lang.String,ontologyFramework.OFContextManagement.OWLReferences,boolean)}{getReasoner}\label{ontologyFramework.OFContextManagement.OWLLibrary.getReasoner(java.lang.String,ontologyFramework.OFContextManagement.OWLReferences,boolean)}}]
~ It creates and returns a Reasoner instance. The type of 
 reasoner is defined by the reasoner name factory, which could be:
 \noprint{@link:#PELLET_reasonerFactoryQualifier}\texttt{\hyperlink{ontologyFramework.OFContextManagement.OWLLibrary.PELLET_reasonerFactoryQualifier}{PELLET_reasonerFactoryQualifier}}, \noprint{@link:#SNOROCKET_reasonerFactoryQualifier}\texttt{\hyperlink{ontologyFramework.OFContextManagement.OWLLibrary.SNOROCKET_reasonerFactoryQualifier}{SNOROCKET_reasonerFactoryQualifier}},
 \noprint{@link:#HERMIT_reasonerFactoryQualifier}\texttt{\hyperlink{ontologyFramework.OFContextManagement.OWLLibrary.HERMIT_reasonerFactoryQualifier}{HERMIT_reasonerFactoryQualifier}} or \noprint{@link:#FACTPLUSPLUS_reasonerFactoryQualifier}\texttt{\hyperlink{ontologyFramework.OFContextManagement.OWLLibrary.FACTPLUSPLUS_reasonerFactoryQualifier}{FACTPLUSPLUS_reasonerFactoryQualifier}}.
 The created reasoner, will be attached to the ontology references given
 as parameter. If buffering flag is true than the reasoner will update its 
 state only if \verb!reasoner.flush()! is called. Otherwise this reasoner
 will synchronizes itself at any ontological changes. The system 
 will return null if a Reflaction error occurs in instancing the
 class defined by the parameter reasonerFactoryName.
\begin{description}
\item[Parameter] ~
\begin{description}
\item[reasonerFactoryName]
full qualifier to the reasoner factory.
\item[ontoRef]
references to the OWL ontology.
\item[buffering]
flag.
\end{description}
\item[Rückgabewert] 
a new instance to the specified reasoner.
\end{description}
\item[{\ltdHypertarget{ontologyFramework.OFContextManagement.OWLLibrary.getPelletReasoner(ontologyFramework.OFContextManagement.OWLReferences,boolean)}{getPelletReasoner}\label{ontologyFramework.OFContextManagement.OWLLibrary.getPelletReasoner(ontologyFramework.OFContextManagement.OWLReferences,boolean)}}]
~ Returns an instance of the Pellet reasoner. If buffering is true
 than the reasoner is update only when \verb!reasoner.flush()! is called.
 Otherwise returns a reasoner which synchronizes itself at any ontological
 changes.
\begin{description}
\item[Parameter] ~
\begin{description}
\item[ontoRef]
references to the OWL ontology
\item[buffering]
flag
\end{description}
\item[Rückgabewert] 
a new Pellet reasoner instance
\end{description}
\item[{\ltdHypertarget{ontologyFramework.OFContextManagement.OWLLibrary.getSnorocketReasoner(ontologyFramework.OFContextManagement.OWLReferences,boolean)}{getSnorocketReasoner}\label{ontologyFramework.OFContextManagement.OWLLibrary.getSnorocketReasoner(ontologyFramework.OFContextManagement.OWLReferences,boolean)}}]
~ Returns an instance of the Snorocket reasoner. If buffering is true
 than the reasoner is update only when \verb!reasoner.flush()! is called.
 Otherwise returns a reasoner which synchronizes itself at any ontological
 changes.
\begin{description}
\item[Parameter] ~
\begin{description}
\item[ontoRef]
references to the OWL ontology
\item[buffering]
flag
\end{description}
\item[Rückgabewert] 
a new Snorocket reasoner instance
\end{description}
\item[{\ltdHypertarget{ontologyFramework.OFContextManagement.OWLLibrary.getHermitReasoner(ontologyFramework.OFContextManagement.OWLReferences,boolean)}{getHermitReasoner}\label{ontologyFramework.OFContextManagement.OWLLibrary.getHermitReasoner(ontologyFramework.OFContextManagement.OWLReferences,boolean)}}]
~ Returns an instance of the Hermit reasoner. If buffering is true
 than the reasoner is update only when \verb!reasoner.flush()! is called.
 Otherwise returns a reasoner which synchronizes itself at any ontological
 changes.
\begin{description}
\item[Parameter] ~
\begin{description}
\item[ontoRef]
references to the OWL ontology
\item[buffering]
flag
\end{description}
\item[Rückgabewert] 
a new Hermit reasoner instance
\end{description}
\item[{\ltdHypertarget{ontologyFramework.OFContextManagement.OWLLibrary.getFactReasoner(ontologyFramework.OFContextManagement.OWLReferences,boolean)}{getFactReasoner}\label{ontologyFramework.OFContextManagement.OWLLibrary.getFactReasoner(ontologyFramework.OFContextManagement.OWLReferences,boolean)}}]
~ Returns an instance of the Fact++ reasoner. If buffering is true
 than the reasoner is update only when \verb!reasoner.flush()! is called.
 Otherwise returns a reasoner which synchronizes itself at any ontological
 changes.
\begin{description}
\item[Parameter] ~
\begin{description}
\item[ontoRef]
references to the OWL ontology
\item[buffering]
flag
\end{description}
\item[Rückgabewert] 
a new Fact++ reasoner instance
\end{description}
\item[{\ltdHypertarget{ontologyFramework.OFContextManagement.OWLLibrary.printOntonolyOnConsole(ontologyFramework.OFContextManagement.OWLReferences)}{printOntonolyOnConsole}\label{ontologyFramework.OFContextManagement.OWLLibrary.printOntonolyOnConsole(ontologyFramework.OFContextManagement.OWLReferences)}}]
~ It prints the ontology over console using Manchester formatting.
\begin{description}
\item[Parameter] ~
\begin{description}
\item[ontoRef]
reference to an OWL ontology
\end{description}
\item[Exceptions] ~
\begin{description}
\item[OWLOntologyStorageException]

\end{description}
\end{description}
\item[{\ltdHypertarget{ontologyFramework.OFContextManagement.OWLLibrary.synchroniseReasoner(ontologyFramework.OFContextManagement.OWLReferences)}{synchroniseReasoner}\label{ontologyFramework.OFContextManagement.OWLLibrary.synchroniseReasoner(ontologyFramework.OFContextManagement.OWLReferences)}}]
~ If the Ontology is consistent it will synchronize a buffering reasoner
 calling \verb!reasoner.flush()!; if the reasoner has a false buffering 
 flag, than this method has no effects. If an inconsistency error 
 occurs than this method will print over console an explanation of the 
 error. Note that if the ontology is inconsistent than all the methods
 in this class may return a null value.
\begin{description}
\item[Parameter] ~
\begin{description}
\item[ontoRef]
references to an OWL ontology.
\end{description}
\end{description}
\item[{\ltdHypertarget{ontologyFramework.OFContextManagement.OWLLibrary.getOWLClass(java.lang.String,ontologyFramework.OFContextManagement.OWLReferences)}{getOWLClass}\label{ontologyFramework.OFContextManagement.OWLLibrary.getOWLClass(java.lang.String,ontologyFramework.OFContextManagement.OWLReferences)}}]
~ Returns an Object which represents an ontological class
 with a given name and specifics IRI paths. If the entity
 already exists in the entology than the object will refer to it, 
 otherwise the method will create a new ontological entity.
\begin{description}
\item[Parameter] ~
\begin{description}
\item[className]
string to define the name of the ontological class
\item[ontoRef]
reference to an OWL ontology.
\end{description}
\item[Rückgabewert] 
the OWL class with the given name and IRI paths in accord to the OWLReference
\end{description}
\item[{\ltdHypertarget{ontologyFramework.OFContextManagement.OWLLibrary.getOWLIndividual(java.lang.String,ontologyFramework.OFContextManagement.OWLReferences)}{getOWLIndividual}\label{ontologyFramework.OFContextManagement.OWLLibrary.getOWLIndividual(java.lang.String,ontologyFramework.OFContextManagement.OWLReferences)}}]
~ Returns an Object which represents an onological individual
 with a given name and specific IRI paths. If the entity
 already exists in the entology than the object will refer to it, 
 otherwise the method will create a new ontological entity.
\begin{description}
\item[Parameter] ~
\begin{description}
\item[individualName]
string to define the name of the ontological individual
\item[ontoRef]
reference to an OWL ontology.
\end{description}
\item[Rückgabewert] 
the OWL individual with the given name and IRI paths in accord to the OWLReference
\end{description}
\item[{\ltdHypertarget{ontologyFramework.OFContextManagement.OWLLibrary.getOWLDataProperty(java.lang.String,ontologyFramework.OFContextManagement.OWLReferences)}{getOWLDataProperty}\label{ontologyFramework.OFContextManagement.OWLLibrary.getOWLDataProperty(java.lang.String,ontologyFramework.OFContextManagement.OWLReferences)}}]
~ Returns an Object which represents an onological data property
 with a given name and specific IRI paths. If the entity
 already exists in the entology than the object will refer to it, 
 otherwise the method will create a new ontological entity.
\begin{description}
\item[Parameter] ~
\begin{description}
\item[dataPropertyName]
string to define the name of the ontological data property
\item[ontoRef]
reference to an OWL ontology.
\end{description}
\item[Rückgabewert] 
the OWL data property with the given name and IRI paths in accord to the OWLReference
\end{description}
\item[{\ltdHypertarget{ontologyFramework.OFContextManagement.OWLLibrary.getOWLObjectProperty(java.lang.String,ontologyFramework.OFContextManagement.OWLReferences)}{getOWLObjectProperty}\label{ontologyFramework.OFContextManagement.OWLLibrary.getOWLObjectProperty(java.lang.String,ontologyFramework.OFContextManagement.OWLReferences)}}]
~ Returns an Object which represents an onological object property
 with a given name and specific IRI paths. If the entity
 already exists in the entology than the object will refer to it, 
 otherwise the method will create a new ontological entity.
\begin{description}
\item[Parameter] ~
\begin{description}
\item[objPropertyName]
string to define the name of the ontological object property
\item[ontoRef]
reference to an OWL ontology.
\end{description}
\item[Rückgabewert] 
the OWL object property with the given name and IRI paths in accord to the OWLReference
\end{description}
\item[{\ltdHypertarget{ontologyFramework.OFContextManagement.OWLLibrary.getOWLLiteral(java.lang.Object,ontologyFramework.OFContextManagement.OWLReferences)}{getOWLLiteral}\label{ontologyFramework.OFContextManagement.OWLLibrary.getOWLLiteral(java.lang.Object,ontologyFramework.OFContextManagement.OWLReferences)}}]
~ Returns an Object which represents an onological literal
 with a given value and specific IRI paths. Indeed it calls:
 \verb!OWLLibrary.getOWLLiteral( value, null, ontoRef)!.
\begin{description}
\item[Parameter] ~
\begin{description}
\item[value]
object to define the value of the ontological literal
\item[ontoRef]
reference to an OWL ontology.
\end{description}
\item[Rückgabewert] 
the OWL literal with the given value, type and IRI paths in accord to the OWLReference
\end{description}
\item[{\ltdHypertarget{ontologyFramework.OFContextManagement.OWLLibrary.getOWLLiteral(java.lang.Object,org.semanticweb.owlapi.model.OWLDatatype,ontologyFramework.OFContextManagement.OWLReferences)}{getOWLLiteral}\label{ontologyFramework.OFContextManagement.OWLLibrary.getOWLLiteral(java.lang.Object,org.semanticweb.owlapi.model.OWLDatatype,ontologyFramework.OFContextManagement.OWLReferences)}}]
~ Given an Object value this method returns the OWLLiteral in accord with the
 actual type of value. The parameter Type can be null if value is of type:
 String, Integer, Boolean, Float, Long; otherwise this method will returns null.
 For more specific data type this methods require to give in input the 
 rigth OWLDataType parameter. Generally it will return null if the 
 data type of the parameter value is unknown.
\begin{description}
\item[Parameter] ~
\begin{description}
\item[value]
object to define the value of the ontological literal
\item[type]
the OWL data type to define the literal
\item[ontoRef]
reference to an OWL ontology.
\end{description}
\item[Rückgabewert] 
the OWL literal with the given value, type and IRI paths in accord to the OWLReference
\end{description}
\item[{\ltdHypertarget{ontologyFramework.OFContextManagement.OWLLibrary.getIndividualB2Class(java.lang.String,ontologyFramework.OFContextManagement.OWLReferences)}{getIndividualB2Class}\label{ontologyFramework.OFContextManagement.OWLLibrary.getIndividualB2Class(java.lang.String,ontologyFramework.OFContextManagement.OWLReferences)}}]
~ It returns all the ontological individual which are defined in the 
 refereed ontology and which are belonging to the calss with name 
 defined by the parameter. Indeed this method will call \noprint{@link:#getOWLClass(String, OWLReferences)}\texttt{\hyperlink{ontologyFramework.OFContextManagement.OWLLibrary.getOWLClass(java.lang.String,ontologyFramework.OFContextManagement.OWLReferences)}{getOWLClass}},
 to get the actual OWL class Object and than it use it to call
 \noprint{@link:#getIndividualB2Class(OWLClass, OWLReferences)}\texttt{\hyperlink{ontologyFramework.OFContextManagement.OWLLibrary.getIndividualB2Class(org.semanticweb.owlapi.model.OWLClass,ontologyFramework.OFContextManagement.OWLReferences)}{getIndividualB2Class}}. Than the 
 returning value is propagated, so it returns null if no individual are
 classified in that class or if such class does not exist in 
 the refereed ontology.
\begin{description}
\item[Parameter] ~
\begin{description}
\item[className]
name of the ontological calss
\item[ontoRef]
reference to an OWL ontology.
\end{description}
\item[Rückgabewert] 
an not ordered set of individual belong to such class.
\end{description}
\item[{\ltdHypertarget{ontologyFramework.OFContextManagement.OWLLibrary.getIndividualB2Class(org.semanticweb.owlapi.model.OWLClass,ontologyFramework.OFContextManagement.OWLReferences)}{getIndividualB2Class}\label{ontologyFramework.OFContextManagement.OWLLibrary.getIndividualB2Class(org.semanticweb.owlapi.model.OWLClass,ontologyFramework.OFContextManagement.OWLReferences)}}]
~ It returns all the ontological individual which are defined in the 
 refereed ontology and which are belonging to the calss 
 defined by the parameter. It returns null if no individual are
 classified in that class or if such class does not exist in 
 the refereed ontology.
\begin{description}
\item[Parameter] ~
\begin{description}
\item[ontoClass]
OWL class for which the individual are asked.
\item[ontoRef]
reference to an OWL ontology.
\end{description}
\item[Rückgabewert] 
an not ordered set of individual belong to such class.
\end{description}
\item[{\ltdHypertarget{ontologyFramework.OFContextManagement.OWLLibrary.getOnlyIndividualB2Class(java.lang.String,ontologyFramework.OFContextManagement.OWLReferences)}{getOnlyIndividualB2Class}\label{ontologyFramework.OFContextManagement.OWLLibrary.getOnlyIndividualB2Class(java.lang.String,ontologyFramework.OFContextManagement.OWLReferences)}}]
~ It returns one ontological individual which are defined in the 
 refereed ontology and which are belonging to the calss with name 
 defined by the parameter. Indeed this method will call \noprint{@link:#getOWLClass(String, OWLReferences)}\texttt{\hyperlink{ontologyFramework.OFContextManagement.OWLLibrary.getOWLClass(java.lang.String,ontologyFramework.OFContextManagement.OWLReferences)}{getOWLClass}},
 to get the actual OWL class Object and than it use it to call
 \noprint{@link:#getIndividualB2Class(OWLClass, OWLReferences)}\texttt{\hyperlink{ontologyFramework.OFContextManagement.OWLLibrary.getIndividualB2Class(org.semanticweb.owlapi.model.OWLClass,ontologyFramework.OFContextManagement.OWLReferences)}{getIndividualB2Class}}. Than,
 using \noprint{@link:#getOnlyElement(Set)}\texttt{\hyperlink{ontologyFramework.OFContextManagement.OWLLibrary.getOnlyElement(java.util.Set<?>)}{getOnlyElement}} it will return one
 individual that are belongign to the class. It returns null if no individual are
 classified in that class, if such class does not exist in 
 the refereed ontology or if the individual set returned by 
 \verb!OWLLibrary.getIndividualB2Class( .. )! has \verb!size > 1!.
\begin{description}
\item[Parameter] ~
\begin{description}
\item[className]
name of the ontological calss
\item[ontoRef]
reference to an OWL ontology.
\end{description}
\item[Rückgabewert] 
an individual belong to such class.
\end{description}
\item[{\ltdHypertarget{ontologyFramework.OFContextManagement.OWLLibrary.getOnlyIndividualB2Class(org.semanticweb.owlapi.model.OWLClass,ontologyFramework.OFContextManagement.OWLReferences)}{getOnlyIndividualB2Class}\label{ontologyFramework.OFContextManagement.OWLLibrary.getOnlyIndividualB2Class(org.semanticweb.owlapi.model.OWLClass,ontologyFramework.OFContextManagement.OWLReferences)}}]
~ It returns an ontological individual which are defined in the 
 refereed ontology and which are belonging to the calss 
 defined by the parameter. It returns null if no individual are
 classified in that class, if such class does not exist in 
 the refereed ontology or if there are more than one
 individual classified in that class (since it uses \noprint{@link:#getOnlyElement(Set)}\texttt{\hyperlink{ontologyFramework.OFContextManagement.OWLLibrary.getOnlyElement(java.util.Set<?>)}{getOnlyElement}}).
\begin{description}
\item[Parameter] ~
\begin{description}
\item[ontoClass]
OWL class for which the individual are asked.
\item[ontoRef]
reference to an OWL ontology.
\end{description}
\item[Rückgabewert] 
an individual belong to such class.
\end{description}
\item[{\ltdHypertarget{ontologyFramework.OFContextManagement.OWLLibrary.getIndividualClasses(org.semanticweb.owlapi.model.OWLNamedIndividual,ontologyFramework.OFContextManagement.OWLReferences)}{getIndividualClasses}\label{ontologyFramework.OFContextManagement.OWLLibrary.getIndividualClasses(org.semanticweb.owlapi.model.OWLNamedIndividual,ontologyFramework.OFContextManagement.OWLReferences)}}]
~ It returns the set of classes in which an individual has been
 classified.
\begin{description}
\item[Parameter] ~
\begin{description}
\item[individual]
ontological individual object
\item[ontoRef]
reference to an OWL ontology.
\end{description}
\item[Rückgabewert] 
a not ordered set of all the classes where the individual is belonging to.
\end{description}
\item[{\ltdHypertarget{ontologyFramework.OFContextManagement.OWLLibrary.getDataPropertyB2Individual(java.lang.String,java.lang.String,ontologyFramework.OFContextManagement.OWLReferences)}{getDataPropertyB2Individual}\label{ontologyFramework.OFContextManagement.OWLLibrary.getDataPropertyB2Individual(java.lang.String,java.lang.String,ontologyFramework.OFContextManagement.OWLReferences)}}]
~ Returns the set of literal value relate to an OWL Data Property 
 which has a specific name and which is assign to a given individual. 
 Indeed it retrieves OWL object from strings and calls: 
 \noprint{@link:#getDataPropertyB2Individual(OWLNamedIndividual, OWLDataProperty, OWLReferences)}\texttt{\hyperlink{ontologyFramework.OFContextManagement.OWLLibrary.getDataPropertyB2Individual(org.semanticweb.owlapi.model.OWLNamedIndividual,org.semanticweb.owlapi.model.OWLDataProperty,ontologyFramework.OFContextManagement.OWLReferences)}{getDataPropertyB2Individual}}.
 Than its returning value is propagated.
\begin{description}
\item[Parameter] ~
\begin{description}
\item[individualName]
name to the ontological individual belonging to the refering ontology
\item[propertyName]
data property name applied to the ontological individual belonging to the refering ontology
\item[ontoRef]
reference to an OWL ontology.
\end{description}
\item[Rückgabewert] 
a not ordered set of literal value of such property applied to a given individual
\end{description}
\item[{\ltdHypertarget{ontologyFramework.OFContextManagement.OWLLibrary.getDataPropertyB2Individual(org.semanticweb.owlapi.model.OWLNamedIndividual,org.semanticweb.owlapi.model.OWLDataProperty,ontologyFramework.OFContextManagement.OWLReferences)}{getDataPropertyB2Individual}\label{ontologyFramework.OFContextManagement.OWLLibrary.getDataPropertyB2Individual(org.semanticweb.owlapi.model.OWLNamedIndividual,org.semanticweb.owlapi.model.OWLDataProperty,ontologyFramework.OFContextManagement.OWLReferences)}}]
~ Returns the set of literal value relate to an OWL Data Property 
 and assigned to a given individual. It returns null if such data property or
 individual doesn not exist. Also if the individual has not such
 proprerty.
\begin{description}
\item[Parameter] ~
\begin{description}
\item[individual]
the OWL individual belonging to the refering ontology
\item[property]
the OWL data property applied to the ontological individual belonging to the refering ontology
\item[ontoRef]
reference to an OWL ontology.
\end{description}
\item[Rückgabewert] 
a not ordered set of literal value of such property applied to a given individual
\end{description}
\item[{\ltdHypertarget{ontologyFramework.OFContextManagement.OWLLibrary.getOnlyDataPropertyB2Individual(java.lang.String,java.lang.String,ontologyFramework.OFContextManagement.OWLReferences)}{getOnlyDataPropertyB2Individual}\label{ontologyFramework.OFContextManagement.OWLLibrary.getOnlyDataPropertyB2Individual(java.lang.String,java.lang.String,ontologyFramework.OFContextManagement.OWLReferences)}}]
~ Returns one literal value attached to a given individual
 througth a specific data property. Here both, individual and property, are given
 by name, than the system calls \noprint{@link:#getOnlyDataPropertyB2Individual(String, String, OWLReferences)}\texttt{\hyperlink{ontologyFramework.OFContextManagement.OWLLibrary.getOnlyDataPropertyB2Individual(java.lang.String,java.lang.String,ontologyFramework.OFContextManagement.OWLReferences)}{getOnlyDataPropertyB2Individual}}
 and its returning value is used with \noprint{@link:#getOnlyElement(Set)}\texttt{\hyperlink{ontologyFramework.OFContextManagement.OWLLibrary.getOnlyElement(java.util.Set<?>)}{getOnlyElement}}.
\begin{description}
\item[Parameter] ~
\begin{description}
\item[individualName]
name to the ontological individual belonging to the refering ontology
\item[propertyName]
data property name applied to the ontological individual belonging to the refering ontology
\item[ontoRef]
reference to an OWL ontology.
\end{description}
\item[Rückgabewert] 
a literal value of such property applied to a given individual
\end{description}
\item[{\ltdHypertarget{ontologyFramework.OFContextManagement.OWLLibrary.getOnlyDataPropertyB2Individual(org.semanticweb.owlapi.model.OWLNamedIndividual,org.semanticweb.owlapi.model.OWLDataProperty,ontologyFramework.OFContextManagement.OWLReferences)}{getOnlyDataPropertyB2Individual}\label{ontologyFramework.OFContextManagement.OWLLibrary.getOnlyDataPropertyB2Individual(org.semanticweb.owlapi.model.OWLNamedIndividual,org.semanticweb.owlapi.model.OWLDataProperty,ontologyFramework.OFContextManagement.OWLReferences)}}]
~ Returns one litteral value attached to a given OWL individual 
 througth an OWL data property. This returns null if \noprint{@link:#getDataPropertyB2Individual(OWLNamedIndividual, OWLDataProperty, OWLReferences)}\texttt{\hyperlink{ontologyFramework.OFContextManagement.OWLLibrary.getDataPropertyB2Individual(org.semanticweb.owlapi.model.OWLNamedIndividual,org.semanticweb.owlapi.model.OWLDataProperty,ontologyFramework.OFContextManagement.OWLReferences)}{getDataPropertyB2Individual}}
 or \noprint{@link:#getOnlyElement(Set)}\texttt{\hyperlink{ontologyFramework.OFContextManagement.OWLLibrary.getOnlyElement(java.util.Set<?>)}{getOnlyElement}} return null.
\begin{description}
\item[Parameter] ~
\begin{description}
\item[individual]
the OWL individual belonging to the refering ontology
\item[property]
the OWL data property applied to the ontological individual belonging to the refering ontology
\item[ontoRef]
reference to an OWL ontology.
\end{description}
\item[Rückgabewert] 
a literal value of such property applied to a given individual
\end{description}
\item[{\ltdHypertarget{ontologyFramework.OFContextManagement.OWLLibrary.getObjectPropertyB2Individual(java.lang.String,java.lang.String,ontologyFramework.OFContextManagement.OWLReferences)}{getObjectPropertyB2Individual}\label{ontologyFramework.OFContextManagement.OWLLibrary.getObjectPropertyB2Individual(java.lang.String,java.lang.String,ontologyFramework.OFContextManagement.OWLReferences)}}]
~ Returns all the values (individuals) to an Object property, given by name,
 linked to an individual, given by name as well. Indeed it retrueve the OWL
 Objects by name using \noprint{@link:#getOWLObjectProperty(String, OWLReferences)}\texttt{\hyperlink{ontologyFramework.OFContextManagement.OWLLibrary.getOWLObjectProperty(java.lang.String,ontologyFramework.OFContextManagement.OWLReferences)}{getOWLObjectProperty}}
 and \noprint{@link:#getOWLIndividual(String, OWLReferences)}\texttt{\hyperlink{ontologyFramework.OFContextManagement.OWLLibrary.getOWLIndividual(java.lang.String,ontologyFramework.OFContextManagement.OWLReferences)}{getOWLIndividual}}. 
 Than it calls \noprint{@link:#getObjectPropertyB2Individual(OWLNamedIndividual, OWLObjectProperty, OWLReferences)}\texttt{\hyperlink{ontologyFramework.OFContextManagement.OWLLibrary.getObjectPropertyB2Individual(org.semanticweb.owlapi.model.OWLNamedIndividual,org.semanticweb.owlapi.model.OWLObjectProperty,ontologyFramework.OFContextManagement.OWLReferences)}{getObjectPropertyB2Individual}}
 propagating its returning value.
\begin{description}
\item[Parameter] ~
\begin{description}
\item[individualName]
the name of an ontological individual
\item[propertyName]
the name of an ontological object property
\item[ontoRef]
reference to an OWL ontology.
\end{description}
\item[Rückgabewert] 
a not ordered set of all the values (OWLNamedIndividual) that
 the individual has w.r.t. such object property.
\end{description}
\item[{\ltdHypertarget{ontologyFramework.OFContextManagement.OWLLibrary.getObjectPropertyB2Individual(org.semanticweb.owlapi.model.OWLNamedIndividual,org.semanticweb.owlapi.model.OWLObjectProperty,ontologyFramework.OFContextManagement.OWLReferences)}{getObjectPropertyB2Individual}\label{ontologyFramework.OFContextManagement.OWLLibrary.getObjectPropertyB2Individual(org.semanticweb.owlapi.model.OWLNamedIndividual,org.semanticweb.owlapi.model.OWLObjectProperty,ontologyFramework.OFContextManagement.OWLReferences)}}]
~ Returns all the values (individuals) to an Object property, given by name,
 linked to an individual, given by name as well. It will return null
 if such object property or individual does not exist.
\begin{description}
\item[Parameter] ~
\begin{description}
\item[individual]
an OWL individual
\item[property]
an OWL object property
\item[ontoRef]
reference to an OWL ontology.
\end{description}
\item[Rückgabewert] 
a not ordered set of all the values (OWLNamedIndividual) that
 the individual has w.r.t. such object property.
\end{description}
\item[{\ltdHypertarget{ontologyFramework.OFContextManagement.OWLLibrary.getOnlyObjectPropertyB2Individual(java.lang.String,java.lang.String,ontologyFramework.OFContextManagement.OWLReferences)}{getOnlyObjectPropertyB2Individual}\label{ontologyFramework.OFContextManagement.OWLLibrary.getOnlyObjectPropertyB2Individual(java.lang.String,java.lang.String,ontologyFramework.OFContextManagement.OWLReferences)}}]
~ Returns a value (individual) to an Object property, given by name,
 linked to an individual, given by name as well. Indeed it retrueve the OWL
 Objects by name using \noprint{@link:#getOWLObjectProperty(String, OWLReferences)}\texttt{\hyperlink{ontologyFramework.OFContextManagement.OWLLibrary-class}{OWLLibrary}}
 and \noprint{@link:#getOWLIndividual(String, OWLReferences)}\texttt{\hyperlink{ontologyFramework.OFContextManagement.OWLLibrary.getOWLIndividual(java.lang.String,ontologyFramework.OFContextManagement.OWLReferences)}{getOWLIndividual}}. 
 Than it calls \noprint{@link:#getObjectPropertyB2Individual(OWLNamedIndividual, OWLObjectProperty, OWLReferences)}\texttt{\hyperlink{ontologyFramework.OFContextManagement.OWLLibrary.getObjectPropertyB2Individual(org.semanticweb.owlapi.model.OWLNamedIndividual,org.semanticweb.owlapi.model.OWLObjectProperty,ontologyFramework.OFContextManagement.OWLReferences)}{getObjectPropertyB2Individual}}
 and its returning value is used to call 
 \noprint{@link:#getOnlyElement(Set)}\texttt{\hyperlink{ontologyFramework.OFContextManagement.OWLLibrary.getOnlyElement(java.util.Set<?>)}{getOnlyElement}} which define the actual returning
 value of this method.
\begin{description}
\item[Parameter] ~
\begin{description}
\item[individualName]
the name of an ontological individual
\item[propertyName]
the name of an ontological object property
\item[ontoRef]
reference to an OWL ontology.
\end{description}
\item[Rückgabewert] 
a value (OWLNamedIndividual) that
 the individual has w.r.t. such object property.
\end{description}
\item[{\ltdHypertarget{ontologyFramework.OFContextManagement.OWLLibrary.getOnlyObjectPropertyB2Individual(org.semanticweb.owlapi.model.OWLNamedIndividual,org.semanticweb.owlapi.model.OWLObjectProperty,ontologyFramework.OFContextManagement.OWLReferences)}{getOnlyObjectPropertyB2Individual}\label{ontologyFramework.OFContextManagement.OWLLibrary.getOnlyObjectPropertyB2Individual(org.semanticweb.owlapi.model.OWLNamedIndividual,org.semanticweb.owlapi.model.OWLObjectProperty,ontologyFramework.OFContextManagement.OWLReferences)}}]
~ Returns a value (individual) to an Object property, given by name,
 linked to an individual, given by name as well. It will return null
 if such object property or individual does not exist. 
 Finally it can return null if \noprint{@link:#getOnlyElement(Set)}\texttt{\hyperlink{ontologyFramework.OFContextManagement.OWLLibrary.getOnlyElement(java.util.Set<?>)}{getOnlyElement}} returns
 null.
\begin{description}
\item[Parameter] ~
\begin{description}
\item[individual]
an OWL individual
\item[property]
an OWL object property
\item[ontoRef]
reference to an OWL ontology.
\end{description}
\item[Rückgabewert] 
a value (OWLNamedIndividual) that
 the individual has w.r.t. such object property.
\end{description}
\item[{\ltdHypertarget{ontologyFramework.OFContextManagement.OWLLibrary.getSubClassOf(java.lang.String,ontologyFramework.OFContextManagement.OWLReferences)}{getSubClassOf}\label{ontologyFramework.OFContextManagement.OWLLibrary.getSubClassOf(java.lang.String,ontologyFramework.OFContextManagement.OWLReferences)}}]
~ Returns all the classes that are sub classes of the given parameter.
 Here class is defined by name, so this method uses: 
 \noprint{@link:#getOWLClass(String, OWLReferences)}\texttt{\hyperlink{ontologyFramework.OFContextManagement.OWLLibrary.getOWLClass(java.lang.String,ontologyFramework.OFContextManagement.OWLReferences)}{getOWLClass}} to get an OWLClass and than
 it calls \noprint{@link:#getSubClassOf(OWLClass, OWLReferences)}\texttt{\hyperlink{ontologyFramework.OFContextManagement.OWLLibrary.getSubClassOf(org.semanticweb.owlapi.model.OWLClass,ontologyFramework.OFContextManagement.OWLReferences)}{getSubClassOf}}
 propagating its returning value.
\begin{description}
\item[Parameter] ~
\begin{description}
\item[className]
name of the ontological class to find sub classes
\item[ontoRef]
reference to an OWL ontology.
\end{description}
\item[Rückgabewert] 
a not order set of all the sub classes of cl parameter.
\end{description}
\item[{\ltdHypertarget{ontologyFramework.OFContextManagement.OWLLibrary.getSubClassOf(org.semanticweb.owlapi.model.OWLClass,ontologyFramework.OFContextManagement.OWLReferences)}{getSubClassOf}\label{ontologyFramework.OFContextManagement.OWLLibrary.getSubClassOf(org.semanticweb.owlapi.model.OWLClass,ontologyFramework.OFContextManagement.OWLReferences)}}]
~ Returns all the classes that are sub classes of the given class parameter.
 It returns null if no sub classes are defined in the ontology.
\begin{description}
\item[Parameter] ~
\begin{description}
\item[cl]
OWL class to find sub classes
\item[ontoRef]
reference to an OWL ontology.
\end{description}
\item[Rückgabewert] 
a not order set of all the sub-classes of cl parameter.
\end{description}
\item[{\ltdHypertarget{ontologyFramework.OFContextManagement.OWLLibrary.getSuperClassOf(java.lang.String,ontologyFramework.OFContextManagement.OWLReferences)}{getSuperClassOf}\label{ontologyFramework.OFContextManagement.OWLLibrary.getSuperClassOf(java.lang.String,ontologyFramework.OFContextManagement.OWLReferences)}}]
~ Returns all the classes that are super classes of the given parameter.
 Here class is defined by name, so this method uses: 
 \noprint{@link:#getOWLClass(String, OWLReferences)}\texttt{\hyperlink{ontologyFramework.OFContextManagement.OWLLibrary.getOWLClass(java.lang.String,ontologyFramework.OFContextManagement.OWLReferences)}{getOWLClass}} to get an OWLClass and than
 it calls \noprint{@link:#getSuperClassOf(OWLClass, OWLReferences)}\texttt{\hyperlink{ontologyFramework.OFContextManagement.OWLLibrary.getSuperClassOf(org.semanticweb.owlapi.model.OWLClass,ontologyFramework.OFContextManagement.OWLReferences)}{getSuperClassOf}}
 propagating its returning value.
\begin{description}
\item[Parameter] ~
\begin{description}
\item[className]
name of the ontological class to find super classes
\item[ontoRef]
reference to an OWL ontology.
\end{description}
\item[Rückgabewert] 
a not order set of all the super classes of cl parameter.
\end{description}
\item[{\ltdHypertarget{ontologyFramework.OFContextManagement.OWLLibrary.getSuperClassOf(org.semanticweb.owlapi.model.OWLClass,ontologyFramework.OFContextManagement.OWLReferences)}{getSuperClassOf}\label{ontologyFramework.OFContextManagement.OWLLibrary.getSuperClassOf(org.semanticweb.owlapi.model.OWLClass,ontologyFramework.OFContextManagement.OWLReferences)}}]
~ Returns all the classes that are super classes of the given class parameter.
 It returns null if no super classes are defined in the ontology.
\begin{description}
\item[Parameter] ~
\begin{description}
\item[cl]
OWL class to find super classes
\item[ontoRef]
reference to an OWL ontology.
\end{description}
\item[Rückgabewert] 
a not order set of all the super classes of cl parameter.
\end{description}
\item[{\ltdHypertarget{ontologyFramework.OFContextManagement.OWLLibrary.setSubClassOf(java.lang.String,java.lang.String,ontologyFramework.OFContextManagement.OWLReferences)}{setSubClassOf}\label{ontologyFramework.OFContextManagement.OWLLibrary.setSubClassOf(java.lang.String,java.lang.String,ontologyFramework.OFContextManagement.OWLReferences)}}]
~ Set the parameter subClass to be a sub class of the
 parameter superClass. Here classes are given by name, the method
 uses \noprint{@link:#getOWLClass(String, OWLReferences)}\texttt{\hyperlink{ontologyFramework.OFContextManagement.OWLLibrary.getOWLClass(java.lang.String,ontologyFramework.OFContextManagement.OWLReferences)}{getOWLClass}} to cal
 \noprint{@link:#setSubClassOf(OWLClass, OWLClass, OWLReferences)}\texttt{\hyperlink{ontologyFramework.OFContextManagement.OWLLibrary.setSubClassOf(org.semanticweb.owlapi.model.OWLClass,org.semanticweb.owlapi.model.OWLClass,ontologyFramework.OFContextManagement.OWLReferences)}{setSubClassOf}} and 
 propagate its returning value.
\begin{description}
\item[Parameter] ~
\begin{description}
\item[superClassName]
the name of the ontological super class
\item[subClassName]
the name of the ontological sub class
\item[ontoRef]
reference to an OWL ontology.
\end{description}
\item[Rückgabewert] 
an ontologial axiom to describe this hyerarchly dependece between classes.
\end{description}
\item[{\ltdHypertarget{ontologyFramework.OFContextManagement.OWLLibrary.setSubClassOf(java.lang.String,java.lang.String,boolean,ontologyFramework.OFContextManagement.OWLReferences)}{setSubClassOf}\label{ontologyFramework.OFContextManagement.OWLLibrary.setSubClassOf(java.lang.String,java.lang.String,boolean,ontologyFramework.OFContextManagement.OWLReferences)}}]
~ Set the parameter subClass to be a sub class of the
 parameter superClass. Here classes are given by name, the method
 uses \noprint{@link:#getOWLClass(String, OWLReferences)}\texttt{\hyperlink{ontologyFramework.OFContextManagement.OWLLibrary.getOWLClass(java.lang.String,ontologyFramework.OFContextManagement.OWLReferences)}{getOWLClass}} to cal
 \noprint{@link:#setSubClassOf(OWLClass, OWLClass, boolean, OWLReferences)}\texttt{\hyperlink{ontologyFramework.OFContextManagement.OWLLibrary.setSubClassOf(org.semanticweb.owlapi.model.OWLClass,org.semanticweb.owlapi.model.OWLClass,boolean,ontologyFramework.OFContextManagement.OWLReferences)}{setSubClassOf}} and 
 propagate its returning value. If the boolean value addAxiom is true,
 than the axioms to add to describe those dependencies are stored in an
 internal buffer. it will be not added to the buffer if it is false.
\begin{description}
\item[Parameter] ~
\begin{description}
\item[superClassName]
the name of the ontological super class
\item[subClassName]
the name of the ontological sub class
\item[addAxiom]
flag to store the adding axioms into a buffer managed in this class.
\item[ontoRef]
reference to an OWL ontology.
\end{description}
\item[Rückgabewert] 
an ontologial axiom to describe this hyerarchly dependece between classes.
\end{description}
\item[{\ltdHypertarget{ontologyFramework.OFContextManagement.OWLLibrary.setSubClassOf(java.lang.String,java.lang.String,boolean,boolean,ontologyFramework.OFContextManagement.OWLReferences)}{setSubClassOf}\label{ontologyFramework.OFContextManagement.OWLLibrary.setSubClassOf(java.lang.String,java.lang.String,boolean,boolean,ontologyFramework.OFContextManagement.OWLReferences)}}]
~ Set the parameter subClass to be a sub class of the
 parameter superClass. Here classes are given by name, the method
 uses \noprint{@link:#getOWLClass(String, OWLReferences)}\texttt{\hyperlink{ontologyFramework.OFContextManagement.OWLLibrary.getOWLClass(java.lang.String,ontologyFramework.OFContextManagement.OWLReferences)}{getOWLClass}} to cal
 \noprint{@link:#setSubClassOf(OWLClass, OWLClass, boolean, boolean, OWLReferences)}\texttt{\hyperlink{ontologyFramework.OFContextManagement.OWLLibrary.setSubClassOf(org.semanticweb.owlapi.model.OWLClass,org.semanticweb.owlapi.model.OWLClass,boolean,boolean,ontologyFramework.OFContextManagement.OWLReferences)}{setSubClassOf}} and 
 propagate its returning value. If the boolean value addAxiom is true,
 than the axioms to add to describe those dependencies are stored in an
 internal buffer. it will be not added to the buffer if it is false.
 On the other hand if the parameter applyChanges is true than those changes are also
 immidiately apllied, otherwise a call to apply them
 is required.
\begin{description}
\item[Parameter] ~
\begin{description}
\item[superClassName]
the name of the ontological super class
\item[subClassName]
the name of the ontological sub class
\item[addAxiom]
flag to store the adding axioms into a buffer managed in this class.
\item[applyChanges]
flag to decide if applyng those change immediatly or not.
\item[ontoRef]
reference to an OWL ontology.
\end{description}
\item[Rückgabewert] 
an ontologial axiom to describe this hyerarchly dependece between classes.
\end{description}
\item[{\ltdHypertarget{ontologyFramework.OFContextManagement.OWLLibrary.setSubClassOf(org.semanticweb.owlapi.model.OWLClass,org.semanticweb.owlapi.model.OWLClass,ontologyFramework.OFContextManagement.OWLReferences)}{setSubClassOf}\label{ontologyFramework.OFContextManagement.OWLLibrary.setSubClassOf(org.semanticweb.owlapi.model.OWLClass,org.semanticweb.owlapi.model.OWLClass,ontologyFramework.OFContextManagement.OWLReferences)}}]
~ Set the parameter subClass to be a sub class of the
 parameter superClass.
\begin{description}
\item[Parameter] ~
\begin{description}
\item[superClass]
the OWL super class
\item[subClass]
the OWL sub class
\item[ontoRef]
reference to an OWL ontology.
\end{description}
\item[Rückgabewert] 
an ontologial axiom to describe this hyerarchly dependece between classes.
\end{description}
\item[{\ltdHypertarget{ontologyFramework.OFContextManagement.OWLLibrary.setSubClassOf(org.semanticweb.owlapi.model.OWLClass,org.semanticweb.owlapi.model.OWLClass,boolean,ontologyFramework.OFContextManagement.OWLReferences)}{setSubClassOf}\label{ontologyFramework.OFContextManagement.OWLLibrary.setSubClassOf(org.semanticweb.owlapi.model.OWLClass,org.semanticweb.owlapi.model.OWLClass,boolean,ontologyFramework.OFContextManagement.OWLReferences)}}]
~ Set the parameter subClass to be a sub class of the
 parameter superClass. If addAxiom flag is true than, this axioms
 will be stored inside an internal buffer, otherwise no. This is
 done by calling \noprint{@link:#getAddAxiom(OWLAxiom, boolean, OWLReferences)}\texttt{\hyperlink{ontologyFramework.OFContextManagement.OWLLibrary.getAddAxiom(org.semanticweb.owlapi.model.OWLAxiom,boolean,ontologyFramework.OFContextManagement.OWLReferences)}{getAddAxiom}}.
\begin{description}
\item[Parameter] ~
\begin{description}
\item[superClass]
the OWL super class
\item[subClass]
the OWL sub class
\item[addAxiom]
flag to store the adding axioms into a buffer managed in this class.	 * @param ontoRef
\item[ontoRef]
reference to an OWL ontology.
\end{description}
\item[Rückgabewert] 
an ontologial axiom to describe this hyerarchly dependece between classes.
\end{description}
\item[{\ltdHypertarget{ontologyFramework.OFContextManagement.OWLLibrary.setSubClassOf(org.semanticweb.owlapi.model.OWLClass,org.semanticweb.owlapi.model.OWLClass,boolean,boolean,ontologyFramework.OFContextManagement.OWLReferences)}{setSubClassOf}\label{ontologyFramework.OFContextManagement.OWLLibrary.setSubClassOf(org.semanticweb.owlapi.model.OWLClass,org.semanticweb.owlapi.model.OWLClass,boolean,boolean,ontologyFramework.OFContextManagement.OWLReferences)}}]
~ Set the parameter subClass to be a sub class of the
 parameter superClass. If addAxiom flag is true than, this axioms
 will be stored inside an internal buffer, otherwise no. This is
 done by calling \noprint{@link:#getAddAxiom(OWLAxiom, boolean, OWLReferences)}\texttt{\hyperlink{ontologyFramework.OFContextManagement.OWLLibrary.getAddAxiom(org.semanticweb.owlapi.model.OWLAxiom,boolean,ontologyFramework.OFContextManagement.OWLReferences)}{getAddAxiom}}.
 On the other hand if applyChanges id true than the changes are immediately 
 moved into the otology, otherwise no. This is done
 by calling \noprint{@link:#applyChanges(OWLReferences)}\texttt{\hyperlink{ontologyFramework.OFContextManagement.OWLLibrary.applyChanges(ontologyFramework.OFContextManagement.OWLReferences)}{applyChanges}}.
\begin{description}
\item[Parameter] ~
\begin{description}
\item[superClass]
the OWL super class
\item[subClass]
the OWL sub class
\item[addAxiom]
flag to store the adding axioms into a buffer managed in this class.
\item[applyChanges]
flag to decide if applyng those change immediatly or not.
\item[ontoRef]
reference to an OWL ontology.
\end{description}
\item[Rückgabewert] 
an ontologial axiom to describe this hyerarchly dependece between classes.
\end{description}
\item[{\ltdHypertarget{ontologyFramework.OFContextManagement.OWLLibrary.getAddAxiom(org.semanticweb.owlapi.model.OWLAxiom,ontologyFramework.OFContextManagement.OWLReferences)}{getAddAxiom}\label{ontologyFramework.OFContextManagement.OWLLibrary.getAddAxiom(org.semanticweb.owlapi.model.OWLAxiom,ontologyFramework.OFContextManagement.OWLReferences)}}]
~ It returns a list of ontology changes to be done to build a 
 given axiom into the ontology. Indeed it calls:
 \noprint{@link:#getAddAxiom(OWLAxiom, boolean, OWLReferences)}\texttt{\hyperlink{ontologyFramework.OFContextManagement.OWLLibrary.getAddAxiom(org.semanticweb.owlapi.model.OWLAxiom,boolean,ontologyFramework.OFContextManagement.OWLReferences)}{getAddAxiom}} with the 
 flag value always set to \verb!true!.
\begin{description}
\item[Parameter] ~
\begin{description}
\item[axiom]
to describe relationsheps between ontological entities.
\item[ontoRef]
reference to an OWL ontology.
\end{description}
\item[Rückgabewert] 
the order set of changes to build a given axiom.
\end{description}
\item[{\ltdHypertarget{ontologyFramework.OFContextManagement.OWLLibrary.getAddAxiom(org.semanticweb.owlapi.model.OWLAxiom,boolean,ontologyFramework.OFContextManagement.OWLReferences)}{getAddAxiom}\label{ontologyFramework.OFContextManagement.OWLLibrary.getAddAxiom(org.semanticweb.owlapi.model.OWLAxiom,boolean,ontologyFramework.OFContextManagement.OWLReferences)}}]
~ It returns a list of ontology changes to be done to build a 
 given axiom into the ontology. If the flag \verb!addToChangeList! is true
 than those changes will be stored inside an internul buffer, otherwise no.
\begin{description}
\item[Parameter] ~
\begin{description}
\item[axiom]
o describe relationsheps between ontological entities.
\item[addToChangeList]
flag to decide if add them into the internal buffer of changes
\item[ontoRef]
reference to an OWL ontology.
\end{description}
\item[Rückgabewert] 
the order set of changes to build a given axiom.
\end{description}
\item[{\ltdHypertarget{ontologyFramework.OFContextManagement.OWLLibrary.getRemoveAxiom(org.semanticweb.owlapi.model.OWLAxiom,ontologyFramework.OFContextManagement.OWLReferences)}{getRemoveAxiom}\label{ontologyFramework.OFContextManagement.OWLLibrary.getRemoveAxiom(org.semanticweb.owlapi.model.OWLAxiom,ontologyFramework.OFContextManagement.OWLReferences)}}]
~ It returns a list of ontology changes to be done to remove a 
 given axiom from the ontology. Indeed it calls:
 \noprint{@link:#getRemoveAxiom(OWLAxiom, boolean, OWLReferences)}\texttt{\hyperlink{ontologyFramework.OFContextManagement.OWLLibrary.getRemoveAxiom(org.semanticweb.owlapi.model.OWLAxiom,boolean,ontologyFramework.OFContextManagement.OWLReferences)}{getRemoveAxiom}} with the 
 flag value always set to \verb!true!.
\begin{description}
\item[Parameter] ~
\begin{description}
\item[axiom]
to describe relationsheps between ontological entities.
\item[ontoRef]
reference to an OWL ontology.
\end{description}
\item[Rückgabewert] 
the order set of changes to remove a given axiom.
\end{description}
\item[{\ltdHypertarget{ontologyFramework.OFContextManagement.OWLLibrary.getRemoveAxiom(org.semanticweb.owlapi.model.OWLAxiom,boolean,ontologyFramework.OFContextManagement.OWLReferences)}{getRemoveAxiom}\label{ontologyFramework.OFContextManagement.OWLLibrary.getRemoveAxiom(org.semanticweb.owlapi.model.OWLAxiom,boolean,ontologyFramework.OFContextManagement.OWLReferences)}}]
~ It returns a list of ontology changes to be done to remove a 
 given axiom from the ontology. If the flag \verb!addToChangeList! is true
 than those changes will be stored inside an internul buffer, otherwise no.
\begin{description}
\item[Parameter] ~
\begin{description}
\item[axiom]
o describe relationsheps between ontological entities.
\item[addToChangeList]
flag to decide if add them into the internal buffer of changes
\item[ontoRef]
reference to an OWL ontology.
\end{description}
\item[Rückgabewert] 
the order set of changes to remove a given axiom.
\end{description}
\item[{\ltdHypertarget{ontologyFramework.OFContextManagement.OWLLibrary.applyChanges(ontologyFramework.OFContextManagement.OWLReferences)}{applyChanges}\label{ontologyFramework.OFContextManagement.OWLLibrary.applyChanges(ontologyFramework.OFContextManagement.OWLReferences)}}]
~ It applies all the changes and axioms stored in the internal buffer 
 into the ontology. After its work, it will clean up this buffer.
\begin{description}
\item[Parameter] ~
\begin{description}
\item[ontoRef]
reference to an OWL ontology.
\end{description}
\end{description}
\item[{\ltdHypertarget{ontologyFramework.OFContextManagement.OWLLibrary.applyChanges(org.semanticweb.owlapi.model.OWLOntologyChange,ontologyFramework.OFContextManagement.OWLReferences)}{applyChanges}\label{ontologyFramework.OFContextManagement.OWLLibrary.applyChanges(org.semanticweb.owlapi.model.OWLOntologyChange,ontologyFramework.OFContextManagement.OWLReferences)}}]
~ It applies, into the ontology, only the change given as parameter.
\begin{description}
\item[Parameter] ~
\begin{description}
\item[addAxiom]
change to apply in the ontology
\item[ontoRef]
param ontoRef reference to an OWL ontology.
\end{description}
\end{description}
\item[{\ltdHypertarget{ontologyFramework.OFContextManagement.OWLLibrary.applyChanges(java.util.List<org.semanticweb.owlapi.model.OWLOntologyChange>,ontologyFramework.OFContextManagement.OWLReferences)}{applyChanges}\label{ontologyFramework.OFContextManagement.OWLLibrary.applyChanges(java.util.List<org.semanticweb.owlapi.model.OWLOntologyChange>,ontologyFramework.OFContextManagement.OWLReferences)}}]
~ It applies, into the ontology, all the changes given as parameter.
\begin{description}
\item[Parameter] ~
\begin{description}
\item[addAxiom]
list of ontological changes.
\item[ontoRef]
reference to an OWL ontology.
\end{description}
\end{description}
\item[{\ltdHypertarget{ontologyFramework.OFContextManagement.OWLLibrary.getSubObjectProperty(java.lang.String,ontologyFramework.OFContextManagement.OWLReferences)}{getSubObjectProperty}\label{ontologyFramework.OFContextManagement.OWLLibrary.getSubObjectProperty(java.lang.String,ontologyFramework.OFContextManagement.OWLReferences)}}]
~ Return all the sub object property relate do a property
 given as an input parameter. Indeed, it retrieve the object from their
 names using \noprint{@link:#getOWLObjectProperty(String, OWLReferences)}\texttt{\hyperlink{ontologyFramework.OFContextManagement.OWLLibrary.getOWLObjectProperty(java.lang.String,ontologyFramework.OFContextManagement.OWLReferences)}{getOWLObjectProperty}}. Than
 it calls \noprint{@link:#getSubObjectProperty(OWLObjectProperty, OWLReferences)}\texttt{\hyperlink{ontologyFramework.OFContextManagement.OWLLibrary.getSubObjectProperty(org.semanticweb.owlapi.model.OWLObjectProperty,ontologyFramework.OFContextManagement.OWLReferences)}{getSubObjectProperty}}
 and propagate its returning value.
\begin{description}
\item[Parameter] ~
\begin{description}
\item[objectPropName]
the name of the ontological object property to check for its sub property
\item[ontoRef]
reference to an OWL ontology.
\end{description}
\item[Rückgabewert] 
an unordered set of Expression to define this hyererchly relations.
\end{description}
\item[{\ltdHypertarget{ontologyFramework.OFContextManagement.OWLLibrary.getSubObjectProperty(org.semanticweb.owlapi.model.OWLObjectProperty,ontologyFramework.OFContextManagement.OWLReferences)}{getSubObjectProperty}\label{ontologyFramework.OFContextManagement.OWLLibrary.getSubObjectProperty(org.semanticweb.owlapi.model.OWLObjectProperty,ontologyFramework.OFContextManagement.OWLReferences)}}]
~ eturn all the sub object property relate do a property
 given as an input parameter.It can return null if no sub object property
 are defined into the ontology for the input parameter.
\begin{description}
\item[Parameter] ~
\begin{description}
\item[objectProp]
the OWL object property to check for its sub property
\item[ontoRef]
reference to an OWL ontology.
\end{description}
\item[Rückgabewert] 
an unordered set of Expression to define this hyererchly relations.
\end{description}
\item[{\ltdHypertarget{ontologyFramework.OFContextManagement.OWLLibrary.getOnlyElement(java.util.Set<?>)}{getOnlyElement}\label{ontologyFramework.OFContextManagement.OWLLibrary.getOnlyElement(java.util.Set<?>)}}]
~ Its pourpuses is to be used when an entity of the ontology can
 have only one element by construction. In particolar this method
 returns true \verb!if( set.size() > 1)!. Otherwise it will iterate over 
 the set and return just the first value. Note that a set does not
 guarantee that its order is always the same.
\begin{description}
\item[Parameter] ~
\begin{description}
\item[set]
a generic set of object
\end{description}
\item[Rückgabewert] 
an element of the set
\end{description}
\item[{\ltdHypertarget{ontologyFramework.OFContextManagement.OWLLibrary.getOnlyString(java.util.Set<org.semanticweb.owlapi.model.OWLLiteral>)}{getOnlyString}\label{ontologyFramework.OFContextManagement.OWLLibrary.getOnlyString(java.util.Set<org.semanticweb.owlapi.model.OWLLiteral>)}}]
~ Its pourpuses is to be used when an entity of the ontology can
 have only one literal by construction. In particolar this method
 returns true \verb!if( set.size() > 1)!. Otherwise it will iterate over 
 the set and return just the first value. Note that a set does not
 guarantee that its order is always the same.
\begin{description}
\item[Parameter] ~
\begin{description}
\item[set]
set of literals
\end{description}
\item[Rückgabewert] 
an element of the set as a string
\end{description}
\item[{\ltdHypertarget{ontologyFramework.OFContextManagement.OWLLibrary.renameEntity(org.semanticweb.owlapi.model.OWLEntity,org.semanticweb.owlapi.model.IRI,ontologyFramework.OFContextManagement.OWLReferences)}{renameEntity}\label{ontologyFramework.OFContextManagement.OWLLibrary.renameEntity(org.semanticweb.owlapi.model.OWLEntity,org.semanticweb.owlapi.model.IRI,ontologyFramework.OFContextManagement.OWLReferences)}}]
~ It returns the changes that must be done into the ontology to rename 
 an entity, they should be applied by calling 
 \verb!applyChanges(renameChanges, ontoRef)!.
\begin{description}
\item[Parameter] ~
\begin{description}
\item[entity]
ontological object to rename
\item[newIRI]
new name as ontological IRI path
\item[ontoRef]
reference to an OWL ontology.
\end{description}
\item[Rückgabewert] 
the changesa to be apllyed into the ontology to rename an entity with a new IRI.
\end{description}
\item[{\ltdHypertarget{ontologyFramework.OFContextManagement.OWLLibrary.renameEntity(org.semanticweb.owlapi.model.OWLEntity,org.semanticweb.owlapi.model.IRI,boolean,ontologyFramework.OFContextManagement.OWLReferences)}{renameEntity}\label{ontologyFramework.OFContextManagement.OWLLibrary.renameEntity(org.semanticweb.owlapi.model.OWLEntity,org.semanticweb.owlapi.model.IRI,boolean,ontologyFramework.OFContextManagement.OWLReferences)}}]
~ It returns the changes that must be done into the ontology to rename 
 an entity. If the flag appyChanges is true than the entity 
 will be immediately renamed into the ontology.
\begin{description}
\item[Parameter] ~
\begin{description}
\item[entity]
ontological object to rename
\item[newIRI]
new name as ontological IRI path
\item[applyChanges]
flag to apply immediatly those changes,
\item[ontoRef]
reference to an OWL ontology.
\end{description}
\item[Rückgabewert] 
the changesa to be apllyed into the ontology to rename an entity with a new IRI.
\end{description}
\item[{\ltdHypertarget{ontologyFramework.OFContextManagement.OWLLibrary.getOWLObjectName(org.semanticweb.owlapi.model.OWLObject)}{getOWLObjectName}\label{ontologyFramework.OFContextManagement.OWLLibrary.getOWLObjectName(org.semanticweb.owlapi.model.OWLObject)}}]
~ It uses a render defined as \verb!OWLObjectRenderer renderer = new DLSyntaxObjectRenderer();!
 to get the name of an ontological object from its IRI path. 
 It returns null if the input parametere is null.
\begin{description}
\item[Parameter] ~
\begin{description}
\item[o]
the objet for which get the ontological name
\end{description}
\item[Rückgabewert] 
the name of the ontological object given as input parameter.
\end{description}
\item[{\ltdHypertarget{ontologyFramework.OFContextManagement.OWLLibrary.getOWLSetAsString(java.util.Set<org.semanticweb.owlapi.model.OWLObject>)}{getOWLSetAsString}\label{ontologyFramework.OFContextManagement.OWLLibrary.getOWLSetAsString(java.util.Set<org.semanticweb.owlapi.model.OWLObject>)}}]
~ Returns a set of names given a set of ontological objects. It uses a 
 renderer: \verb!OWLObjectRenderer renderer = new DLSyntaxObjectRenderer();!
 to do this work. The inoput set cannot have null value. It will riturn
 null if the input set is empty.
\begin{description}
\item[Parameter] ~
\begin{description}
\item[set]
of ontological object from which retrieve names.
\end{description}
\item[Rückgabewert] 
set of names of the object contained in the input parameter.
\end{description}
\item[{\ltdHypertarget{ontologyFramework.OFContextManagement.OWLLibrary.getOWLLiteralAsString(java.util.Set<org.semanticweb.owlapi.model.OWLLiteral>)}{getOWLLiteralAsString}\label{ontologyFramework.OFContextManagement.OWLLibrary.getOWLLiteralAsString(java.util.Set<org.semanticweb.owlapi.model.OWLLiteral>)}}]
~ It converts all the literal inside a set into a set of string using
 \verb!literal.getLiteral()!. The input set cannot contain null values.
 This method returns null if the input set is empty.
\begin{description}
\item[Parameter] ~
\begin{description}
\item[set]
set of ontological individual.
\end{description}
\item[Rückgabewert] 
the input set converterd to string.
\end{description}
\item[{\ltdHypertarget{ontologyFramework.OFContextManagement.OWLLibrary.saveOntology(boolean,ontologyFramework.OFContextManagement.OWLReferences)}{saveOntology}\label{ontologyFramework.OFContextManagement.OWLLibrary.saveOntology(boolean,ontologyFramework.OFContextManagement.OWLReferences)}}]
~ It will save an ontology into a file. The files path is 
 retrieved from the OWLReferences class using: 
 \verb!ontoRef.getIriFilePath();!. Note that this procedure
 may replace an already existing file. The exporting of the 
 asserted relation is done by: \noprint{@link:InferedAxiomExporter#exportOntology(OWLReferences)}\texttt{\hyperlink{ontologyFramework.OFContextManagement.InferedAxiomExporter.exportOntology(ontologyFramework.OFContextManagement.OWLReferences)}{exportOntology}}
 and my be an expencive procedure.
\begin{description}
\item[Parameter] ~
\begin{description}
\item[exportInf]
flag to export inferences as fixed relations.
\item[ontoRef]
reference to an OWL ontology.
\end{description}
\end{description}
\item[{\ltdHypertarget{ontologyFramework.OFContextManagement.OWLLibrary.saveOntology(boolean,java.lang.String,ontologyFramework.OFContextManagement.OWLReferences)}{saveOntology}\label{ontologyFramework.OFContextManagement.OWLLibrary.saveOntology(boolean,java.lang.String,ontologyFramework.OFContextManagement.OWLReferences)}}]
~ It will save an ontology into a file. The files path is 
 given as input parameter, and this method does not update: 
 \verb!ontoRef.getIriFilePath();!. Note that this procedure
 may replace an already existing file. The exporting of the 
 asserted relation is done by: \noprint{@link:InferedAxiomExporter#exportOntology(OWLReferences)}\texttt{\hyperlink{ontologyFramework.OFContextManagement.InferedAxiomExporter.exportOntology(ontologyFramework.OFContextManagement.OWLReferences)}{exportOntology}}
 and my be an expencive procedure.
\begin{description}
\item[Parameter] ~
\begin{description}
\item[exportInf]
flag to export inferences as fixed relations.
\item[filePath]
directiory in which save the ontology.
\item[ontoRef]
reference to an OWL ontology.
\end{description}
\end{description}
\item[{\ltdHypertarget{ontologyFramework.OFContextManagement.OWLLibrary.addObjectPropertyB2Individual(org.semanticweb.owlapi.model.OWLNamedIndividual,org.semanticweb.owlapi.model.OWLObjectProperty,org.semanticweb.owlapi.model.OWLNamedIndividual,boolean,ontologyFramework.OFContextManagement.OWLReferences)}{addObjectPropertyB2Individual}\label{ontologyFramework.OFContextManagement.OWLLibrary.addObjectPropertyB2Individual(org.semanticweb.owlapi.model.OWLNamedIndividual,org.semanticweb.owlapi.model.OWLObjectProperty,org.semanticweb.owlapi.model.OWLNamedIndividual,boolean,ontologyFramework.OFContextManagement.OWLReferences)}}]
~ Returns a list of changes to be applied into the ontology to
 add a new object property (with its value) into an individual.
 If the bufferize flag is true than those changes will be saved inside at the
 internal buffer of this class which can be applied by calling:
 \noprint{@link:#applyChanges(OWLReferences)}\texttt{\hyperlink{ontologyFramework.OFContextManagement.OWLLibrary.applyChanges(ontologyFramework.OFContextManagement.OWLReferences)}{applyChanges}}. If this flag is false than only this
 changes will be immediately applied to the refering ontology.
\begin{description}
\item[Parameter] ~
\begin{description}
\item[ind]
individual that have to have a new object property.
\item[prop]
object property to be added.
\item[value]
individual which is the value of the given object property.
\item[bufferize]
flag to bufferize changes inside an internal buffer.
\item[ontoRef]
reference to an OWL ontology.
\end{description}
\item[Rückgabewert] 
the changes to be done into the refereed ontology to add this specific object property.
\end{description}
\item[{\ltdHypertarget{ontologyFramework.OFContextManagement.OWLLibrary.addObjectPropertyB2Individual(java.lang.String,java.lang.String,java.lang.String,boolean,ontologyFramework.OFContextManagement.OWLReferences)}{addObjectPropertyB2Individual}\label{ontologyFramework.OFContextManagement.OWLLibrary.addObjectPropertyB2Individual(java.lang.String,java.lang.String,java.lang.String,boolean,ontologyFramework.OFContextManagement.OWLReferences)}}]
~ Returns a list of changes to be applied into the ontology to
 add a new object property (with its value) into an individual.
 If the bufferize flag is true than those changes will be saved inside at the
 internal buffer of this class which can be applied by calling:
 \noprint{@link:#applyChanges(OWLReferences)}\texttt{\hyperlink{ontologyFramework.OFContextManagement.OWLLibrary.applyChanges(ontologyFramework.OFContextManagement.OWLReferences)}{applyChanges}}. If this flag is false than only this
 changes will be immediately applied to the refering ontology.
 Indeed it retrieve the ontological object from name and than it calls: 
 \noprint{@link:#addObjectPropertyB2Individual(OWLNamedIndividual, OWLObjectProperty, OWLNamedIndividual, boolean, OWLReferences)}\texttt{\hyperlink{ontologyFramework.OFContextManagement.OWLLibrary.addObjectPropertyB2Individual(org.semanticweb.owlapi.model.OWLNamedIndividual,org.semanticweb.owlapi.model.OWLObjectProperty,org.semanticweb.owlapi.model.OWLNamedIndividual,boolean,ontologyFramework.OFContextManagement.OWLReferences)}{addObjectPropertyB2Individual}}
\begin{description}
\item[Parameter] ~
\begin{description}
\item[individualName]
tha name of an ontological individual that havo to have a new object property
\item[propName]
name of the object property inside the ontology refered by ontoRef.
\item[valueName]
individual name inside te refereed ontology to be the value of the given object property
\item[bufferize]
flag to buffering changes internally to this class.
\item[ontoRef]
reference to an OWL ontology.
\end{description}
\item[Rückgabewert] 
the changes to be done into the refereed ontology to add this specific object property.
\end{description}
\item[{\ltdHypertarget{ontologyFramework.OFContextManagement.OWLLibrary.addDataPropertyB2Individual(org.semanticweb.owlapi.model.OWLNamedIndividual,org.semanticweb.owlapi.model.OWLDataProperty,org.semanticweb.owlapi.model.OWLLiteral,boolean,ontologyFramework.OFContextManagement.OWLReferences)}{addDataPropertyB2Individual}\label{ontologyFramework.OFContextManagement.OWLLibrary.addDataPropertyB2Individual(org.semanticweb.owlapi.model.OWLNamedIndividual,org.semanticweb.owlapi.model.OWLDataProperty,org.semanticweb.owlapi.model.OWLLiteral,boolean,ontologyFramework.OFContextManagement.OWLReferences)}}]
~ Returns a list of changes to be applied into the ontology to
 add a new datat property (with its value) into an individual.
 If the bufferize flag is true than those changes will be saved inside at the
 internal buffer of this class which can be applied by calling:
 \noprint{@link:#applyChanges(OWLReferences)}\texttt{\hyperlink{ontologyFramework.OFContextManagement.OWLLibrary.applyChanges(ontologyFramework.OFContextManagement.OWLReferences)}{applyChanges}}. If this flag is false than only this
 changes will be immediately applied to the refering ontology.
\begin{description}
\item[Parameter] ~
\begin{description}
\item[ind]
individual that have to have a new data property.
\item[prop]
data property to be added.
\item[value]
literal which is the value of the given data property.
\item[bufferize]
flag to bufferize changes inside an internal buffer.
\item[ontoRef]
reference to an OWL ontology.
\end{description}
\item[Rückgabewert] 
the changes to be done into the refereed ontology to add this specific data property.
\end{description}
\item[{\ltdHypertarget{ontologyFramework.OFContextManagement.OWLLibrary.addDataPropertyB2Individual(java.lang.String,java.lang.String,java.lang.Object,boolean,ontologyFramework.OFContextManagement.OWLReferences)}{addDataPropertyB2Individual}\label{ontologyFramework.OFContextManagement.OWLLibrary.addDataPropertyB2Individual(java.lang.String,java.lang.String,java.lang.Object,boolean,ontologyFramework.OFContextManagement.OWLReferences)}}]
~ Returns a list of changes to be applied into the ontology to
 add a new data property (with its value) into an individual.
 If the bufferize flag is true than those changes will be saved inside at the
 internal buffer of this class which can be applied by calling:
 \noprint{@link:#applyChanges(OWLReferences)}\texttt{\hyperlink{ontologyFramework.OFContextManagement.OWLLibrary.applyChanges(ontologyFramework.OFContextManagement.OWLReferences)}{applyChanges}}. If this flag is false than only this
 changes will be immediately applied to the refering ontology.
 Indeed it retrieve the ontological object from name and than it calls: 
 \noprint{@link:#addDataPropertyB2Individual(OWLNamedIndividual, OWLDataProperty, OWLLiteral, boolean, OWLReferences)}\texttt{\hyperlink{ontologyFramework.OFContextManagement.OWLLibrary.addDataPropertyB2Individual(org.semanticweb.owlapi.model.OWLNamedIndividual,org.semanticweb.owlapi.model.OWLDataProperty,org.semanticweb.owlapi.model.OWLLiteral,boolean,ontologyFramework.OFContextManagement.OWLReferences)}{addDataPropertyB2Individual}}
\begin{description}
\item[Parameter] ~
\begin{description}
\item[individualName]
tha name of an ontological individual that havo to have a new data property
\item[propertyName]
name of the data property inside the ontology refered by ontoRef.
\item[value]
literal to be added as the value of a data property.
\item[bufferize]
flag to buffering changes internally to this class.
\item[ontoRef]
reference to an OWL ontology.
\end{description}
\item[Rückgabewert] 
the changes to be done into the refereed ontology to add the given data property.
\end{description}
\item[{\ltdHypertarget{ontologyFramework.OFContextManagement.OWLLibrary.addIndividualB2Class(org.semanticweb.owlapi.model.OWLNamedIndividual,org.semanticweb.owlapi.model.OWLClass,boolean,ontologyFramework.OFContextManagement.OWLReferences)}{addIndividualB2Class}\label{ontologyFramework.OFContextManagement.OWLLibrary.addIndividualB2Class(org.semanticweb.owlapi.model.OWLNamedIndividual,org.semanticweb.owlapi.model.OWLClass,boolean,ontologyFramework.OFContextManagement.OWLReferences)}}]
~ Returns the ontological changes to be applyed to put an 
 individual inside an ontological class.
 If the bufferize flag is true than those changes will be saved inside at the
 internal buffer of this class which can be applied by calling:
 \noprint{@link:#applyChanges(OWLReferences)}\texttt{\hyperlink{ontologyFramework.OFContextManagement.OWLLibrary.applyChanges(ontologyFramework.OFContextManagement.OWLReferences)}{applyChanges}}. If this flag is false than only this
 changes will be immediately applied to the refering ontology.
\begin{description}
\item[Parameter] ~
\begin{description}
\item[ind]
individual to add into an ontological class
\item[c]
ontological class that than will conteind this individual
\item[bufferize]
flag to bufferize changes inside an internal buffer.
\item[ontoRef]
reference to an OWL ontology.
\end{description}
\item[Rückgabewert] 
changes to be done into the refereed ontology to set an individual to be belonging to a specific class.
\end{description}
\item[{\ltdHypertarget{ontologyFramework.OFContextManagement.OWLLibrary.addIndividualB2Class(java.lang.String,java.lang.String,boolean,ontologyFramework.OFContextManagement.OWLReferences)}{addIndividualB2Class}\label{ontologyFramework.OFContextManagement.OWLLibrary.addIndividualB2Class(java.lang.String,java.lang.String,boolean,ontologyFramework.OFContextManagement.OWLReferences)}}]
~ Returns a list of changes to be applied into the ontology to
 set an individual to belonging to a class.
 If the bufferize flag is true than those changes will be saved inside at the
 internal buffer of this class which can be applied by calling:
 \noprint{@link:#applyChanges(OWLReferences)}\texttt{\hyperlink{ontologyFramework.OFContextManagement.OWLLibrary.applyChanges(ontologyFramework.OFContextManagement.OWLReferences)}{applyChanges}}. If this flag is false than only this
 changes will be immediately applied to the refering ontology.
 Indeed it retrieve the ontological object from name inside the refering ontology
 and than it calls: 
 \noprint{@link:#addIndividualB2Class(OWLNamedIndividual, OWLClass, boolean, OWLReferences)}\texttt{\hyperlink{ontologyFramework.OFContextManagement.OWLLibrary.addIndividualB2Class(org.semanticweb.owlapi.model.OWLNamedIndividual,org.semanticweb.owlapi.model.OWLClass,boolean,ontologyFramework.OFContextManagement.OWLReferences)}{addIndividualB2Class}}
\begin{description}
\item[Parameter] ~
\begin{description}
\item[individualName]
tha name of an ontological individual that have to be beloging to a given class.
\item[className]
the name of an ontological class that will contains the input individual parameter.
\item[bufferize]
flag to buffering changes internally to this class.
\item[ontoRef]
reference to an OWL ontology.
\end{description}
\item[Rückgabewert] 
the changes to be done into the refereed ontology to set an individual belong to a class.
\end{description}
\item[{\ltdHypertarget{ontologyFramework.OFContextManagement.OWLLibrary.removeObjectPropertyB2Individual(org.semanticweb.owlapi.model.OWLNamedIndividual,org.semanticweb.owlapi.model.OWLObjectProperty,org.semanticweb.owlapi.model.OWLNamedIndividual,boolean,ontologyFramework.OFContextManagement.OWLReferences)}{removeObjectPropertyB2Individual}\label{ontologyFramework.OFContextManagement.OWLLibrary.removeObjectPropertyB2Individual(org.semanticweb.owlapi.model.OWLNamedIndividual,org.semanticweb.owlapi.model.OWLObjectProperty,org.semanticweb.owlapi.model.OWLNamedIndividual,boolean,ontologyFramework.OFContextManagement.OWLReferences)}}]
~ Returns a list of changes to be applied into the ontology to
 remove an object property (with its value) from an individual.
 If the bufferize flag is true than those changes will be saved inside at the
 internal buffer of this class which can be applied by calling:
 \noprint{@link:#applyChanges(OWLReferences)}\texttt{\hyperlink{ontologyFramework.OFContextManagement.OWLLibrary.applyChanges(ontologyFramework.OFContextManagement.OWLReferences)}{applyChanges}}. If this flag is false than only this
 changes will be immediately applied to the refering ontology.
\begin{description}
\item[Parameter] ~
\begin{description}
\item[ind]
individual from which remove a given object property.
\item[prop]
object property to be removed.
\item[value]
individual which is the value of the given object property.
\item[bufferize]
flag to bufferize changes inside an internal buffer.
\item[ontoRef]
reference to an OWL ontology.
\end{description}
\item[Rückgabewert] 
the changes to be done into the refereed ontology to remove this specific object property.
\end{description}
\item[{\ltdHypertarget{ontologyFramework.OFContextManagement.OWLLibrary.removeObjectPropertyB2Individual(java.lang.String,java.lang.String,java.lang.String,boolean,ontologyFramework.OFContextManagement.OWLReferences)}{removeObjectPropertyB2Individual}\label{ontologyFramework.OFContextManagement.OWLLibrary.removeObjectPropertyB2Individual(java.lang.String,java.lang.String,java.lang.String,boolean,ontologyFramework.OFContextManagement.OWLReferences)}}]
~ Returns a list of changes to be applied into the ontology to
 remove a given object property (with its value) from an individual.
 If the bufferize flag is true than those changes will be saved inside at the
 internal buffer of this class which can be applied by calling:
 \noprint{@link:#applyChanges(OWLReferences)}\texttt{\hyperlink{ontologyFramework.OFContextManagement.OWLLibrary.applyChanges(ontologyFramework.OFContextManagement.OWLReferences)}{applyChanges}}. If this flag is false than only this
 changes will be immediately applied to the refering ontology.
 Indeed it retrieve the ontological object from name and than it calls: 
 \noprint{@link:#removeObjectPropertyB2Individual(OWLNamedIndividual, OWLObjectProperty, OWLNamedIndividual, boolean, OWLReferences)}\texttt{\hyperlink{ontologyFramework.OFContextManagement.OWLLibrary.removeObjectPropertyB2Individual(org.semanticweb.owlapi.model.OWLNamedIndividual,org.semanticweb.owlapi.model.OWLObjectProperty,org.semanticweb.owlapi.model.OWLNamedIndividual,boolean,ontologyFramework.OFContextManagement.OWLReferences)}{removeObjectPropertyB2Individual}}
\begin{description}
\item[Parameter] ~
\begin{description}
\item[individualName]
tha name of an ontological individual from which remove the object property
\item[propName]
name of the object property inside the ontology refered by ontoRef.
\item[valueName]
individual name inside te refereed ontology to be the value of the given object property
\item[bufferize]
flag to buffering changes internally to this class.
\item[ontoRef]
reference to an OWL ontology.
\end{description}
\item[Rückgabewert] 
the changes to be done into the refereed ontology to remove this specific object property.
\end{description}
\item[{\ltdHypertarget{ontologyFramework.OFContextManagement.OWLLibrary.removeDataPropertyB2Individual(org.semanticweb.owlapi.model.OWLNamedIndividual,org.semanticweb.owlapi.model.OWLDataProperty,org.semanticweb.owlapi.model.OWLLiteral,boolean,ontologyFramework.OFContextManagement.OWLReferences)}{removeDataPropertyB2Individual}\label{ontologyFramework.OFContextManagement.OWLLibrary.removeDataPropertyB2Individual(org.semanticweb.owlapi.model.OWLNamedIndividual,org.semanticweb.owlapi.model.OWLDataProperty,org.semanticweb.owlapi.model.OWLLiteral,boolean,ontologyFramework.OFContextManagement.OWLReferences)}}]
~ Returns a list of changes to be applied into the ontology to
 remove  datat property (with its value) from an individual.
 If the bufferize flag is true than those changes will be saved inside at the
 internal buffer of this class which can be applied by calling:
 \noprint{@link:#applyChanges(OWLReferences)}\texttt{\hyperlink{ontologyFramework.OFContextManagement.OWLLibrary.applyChanges(ontologyFramework.OFContextManagement.OWLReferences)}{applyChanges}}. If this flag is false than only this
 changes will be immediately applied to the refering ontology.
\begin{description}
\item[Parameter] ~
\begin{description}
\item[ind]
individual from which remove the given data property.
\item[prop]
data property to be removed.
\item[value]
literal which is the value of the given data property.
\item[bufferize]
flag to bufferize changes inside an internal buffer.
\item[ontoRef]
reference to an OWL ontology.
\end{description}
\item[Rückgabewert] 
the changes to be done into the refereed ontology to remove this specific data property.
\end{description}
\item[{\ltdHypertarget{ontologyFramework.OFContextManagement.OWLLibrary.removeDataPropertyB2Individual(java.lang.String,java.lang.String,java.lang.Object,boolean,ontologyFramework.OFContextManagement.OWLReferences)}{removeDataPropertyB2Individual}\label{ontologyFramework.OFContextManagement.OWLLibrary.removeDataPropertyB2Individual(java.lang.String,java.lang.String,java.lang.Object,boolean,ontologyFramework.OFContextManagement.OWLReferences)}}]
~ Returns a list of changes to be applied into the ontology to
 remove a data property (with its value) from an individual.
 If the bufferize flag is true than those changes will be saved inside at the
 internal buffer of this class which can be applied by calling:
 \noprint{@link:#applyChanges(OWLReferences)}\texttt{\hyperlink{ontologyFramework.OFContextManagement.OWLLibrary.applyChanges(ontologyFramework.OFContextManagement.OWLReferences)}{applyChanges}}. If this flag is false than only this
 changes will be immediately applied to the refering ontology.
 Indeed it retrieve the ontological object from name and than it calls: 
 \noprint{@link:#removeDataPropertyB2Individual(OWLNamedIndividual, OWLDataProperty, OWLLiteral, boolean, OWLReferences)}\texttt{\hyperlink{ontologyFramework.OFContextManagement.OWLLibrary.removeDataPropertyB2Individual(org.semanticweb.owlapi.model.OWLNamedIndividual,org.semanticweb.owlapi.model.OWLDataProperty,org.semanticweb.owlapi.model.OWLLiteral,boolean,ontologyFramework.OFContextManagement.OWLReferences)}{removeDataPropertyB2Individual}}
\begin{description}
\item[Parameter] ~
\begin{description}
\item[individualName]
tha name of an ontological individual from which remove the data property
\item[propertyName]
name of the data property inside the ontology refered by ontoRef.
\item[value]
literal to be removed as the value of a data property.
\item[bufferize]
flag to buffering changes internally to this class.
\item[ontoRef]
reference to an OWL ontology.
\end{description}
\item[Rückgabewert] 
the changes to be done into the refereed ontology to remove this specific data property.
\end{description}
\item[{\ltdHypertarget{ontologyFramework.OFContextManagement.OWLLibrary.removeIndividualB2Class(org.semanticweb.owlapi.model.OWLNamedIndividual,org.semanticweb.owlapi.model.OWLClass,boolean,ontologyFramework.OFContextManagement.OWLReferences)}{removeIndividualB2Class}\label{ontologyFramework.OFContextManagement.OWLLibrary.removeIndividualB2Class(org.semanticweb.owlapi.model.OWLNamedIndividual,org.semanticweb.owlapi.model.OWLClass,boolean,ontologyFramework.OFContextManagement.OWLReferences)}}]
~ Returns the ontological changes to be applyed to remove an 
 individual from an ontological class.
 If the bufferize flag is true than those changes will be saved inside at the
 internal buffer of this class which can be applied by calling:
 \noprint{@link:#applyChanges(OWLReferences)}\texttt{\hyperlink{ontologyFramework.OFContextManagement.OWLLibrary.applyChanges(ontologyFramework.OFContextManagement.OWLReferences)}{applyChanges}}. If this flag is false than only this
 changes will be immediately applied to the refering ontology.
\begin{description}
\item[Parameter] ~
\begin{description}
\item[ind]
individual to remove from an ontological class
\item[c]
ontological class that than was conteind this individual
\item[bufferize]
flag to bufferize changes inside an internal buffer.
\item[ontoRef]
reference to an OWL ontology.
\end{description}
\item[Rückgabewert] 
changes to be done into the refereed ontology to set an individual to not be anymore belonging to a specific class.
\end{description}
\item[{\ltdHypertarget{ontologyFramework.OFContextManagement.OWLLibrary.removeIndividualB2Class(java.lang.String,java.lang.String,boolean,ontologyFramework.OFContextManagement.OWLReferences)}{removeIndividualB2Class}\label{ontologyFramework.OFContextManagement.OWLLibrary.removeIndividualB2Class(java.lang.String,java.lang.String,boolean,ontologyFramework.OFContextManagement.OWLReferences)}}]
~ Returns a list of changes to be applied into the ontology to
 remove an individual to belonging to a class.
 If the bufferize flag is true than those changes will be saved inside at the
 internal buffer of this class which can be applied by calling:
 \noprint{@link:#applyChanges(OWLReferences)}\texttt{\hyperlink{ontologyFramework.OFContextManagement.OWLLibrary.applyChanges(ontologyFramework.OFContextManagement.OWLReferences)}{applyChanges}}. If this flag is false than only this
 changes will be immediately applied to the refering ontology.
 Indeed it retrieve the ontological object from name inside the refering ontology
 and than it calls: 
 \noprint{@link:#removeIndividualB2Class(OWLNamedIndividual, OWLClass, boolean, OWLReferences)}\texttt{\hyperlink{ontologyFramework.OFContextManagement.OWLLibrary.removeIndividualB2Class(org.semanticweb.owlapi.model.OWLNamedIndividual,org.semanticweb.owlapi.model.OWLClass,boolean,ontologyFramework.OFContextManagement.OWLReferences)}{removeIndividualB2Class}}
\begin{description}
\item[Parameter] ~
\begin{description}
\item[individualName]
tha name of an ontological individual that have not to be beloging to a given class.
\item[className]
the name of an ontological class that will no more contains the input individual parameter.
\item[bufferize]
flag to buffering changes internally to this class.
\item[ontoRef]
reference to an OWL ontology.
\end{description}
\item[Rückgabewert] 
the changes to be done into the refereed ontology to set an individual to do not belong to a class anymore.
\end{description}
\item[{\ltdHypertarget{ontologyFramework.OFContextManagement.OWLLibrary.removeIndividual(org.semanticweb.owlapi.model.OWLNamedIndividual,java.lang.Boolean,ontologyFramework.OFContextManagement.OWLReferences)}{removeIndividual}\label{ontologyFramework.OFContextManagement.OWLLibrary.removeIndividual(org.semanticweb.owlapi.model.OWLNamedIndividual,java.lang.Boolean,ontologyFramework.OFContextManagement.OWLReferences)}}]
~ Returns the changes to be apllied into the refering ontology for 
 removing an individual.
 If the bufferize flag is true than those changes will be saved inside at the
 internal buffer of this class which can be applied by calling:
 \noprint{@link:#applyChanges(OWLReferences)}\texttt{\hyperlink{ontologyFramework.OFContextManagement.OWLLibrary.applyChanges(ontologyFramework.OFContextManagement.OWLReferences)}{applyChanges}}. If this flag is false than only this
 changes will be immediately applied to the refering ontology.
\begin{description}
\item[Parameter] ~
\begin{description}
\item[individual]
to be removed from the ontology.
\item[bufferize]
flag to buffering changes internally to this class.
\item[ontoRef]
reference to an OWL ontology.
\end{description}
\item[Rückgabewert] 
tha changes to be done into the refereed ontology to remove a given individual.
\end{description}
\item[{\ltdHypertarget{ontologyFramework.OFContextManagement.OWLLibrary.removeIndividual(java.util.Set<org.semanticweb.owlapi.model.OWLNamedIndividual>,java.lang.Boolean,ontologyFramework.OFContextManagement.OWLReferences)}{removeIndividual}\label{ontologyFramework.OFContextManagement.OWLLibrary.removeIndividual(java.util.Set<org.semanticweb.owlapi.model.OWLNamedIndividual>,java.lang.Boolean,ontologyFramework.OFContextManagement.OWLReferences)}}]
~ Returns the changes to be applied into the refering ontology for removing
 a set of individuals.
 If the bufferize flag is true than those changes will be saved inside at the
 internal buffer of this class which can be applied by calling:
 \noprint{@link:#applyChanges(OWLReferences)}\texttt{\hyperlink{ontologyFramework.OFContextManagement.OWLLibrary.applyChanges(ontologyFramework.OFContextManagement.OWLReferences)}{applyChanges}}. If this flag is false than only this
 changes will be immediately applied to the refering ontology.
\begin{description}
\item[Parameter] ~
\begin{description}
\item[individuals]
set of individuals to be removed.
\item[bufferised]
flag to buffering changes internally to this class.
\item[ontoRef]
reference to an OWL ontology.
\end{description}
\item[Rückgabewert] 
tha changes to be done into the refereed ontology to remove a given set individuals.
\end{description}
\item[{\ltdHypertarget{ontologyFramework.OFContextManagement.OWLLibrary.replaceDataProperty(org.semanticweb.owlapi.model.OWLNamedIndividual,org.semanticweb.owlapi.model.OWLDataProperty,java.util.Set<org.semanticweb.owlapi.model.OWLLiteral>,org.semanticweb.owlapi.model.OWLLiteral,java.lang.Boolean,ontologyFramework.OFContextManagement.OWLReferences)}{replaceDataProperty}\label{ontologyFramework.OFContextManagement.OWLLibrary.replaceDataProperty(org.semanticweb.owlapi.model.OWLNamedIndividual,org.semanticweb.owlapi.model.OWLDataProperty,java.util.Set<org.semanticweb.owlapi.model.OWLLiteral>,org.semanticweb.owlapi.model.OWLLiteral,java.lang.Boolean,ontologyFramework.OFContextManagement.OWLReferences)}}]
~ Atomically (with respect to reasoner update) replacing of a data property.
 Indeed, it will remove all the possible data property with a given values
 using \noprint{@link:#removeDataPropertyB2Individual(OWLNamedIndividual, OWLDataProperty, OWLLiteral, boolean, OWLReferences)}\texttt{\hyperlink{ontologyFramework.OFContextManagement.OWLLibrary.removeDataPropertyB2Individual(org.semanticweb.owlapi.model.OWLNamedIndividual,org.semanticweb.owlapi.model.OWLDataProperty,org.semanticweb.owlapi.model.OWLLiteral,boolean,ontologyFramework.OFContextManagement.OWLReferences)}{removeDataPropertyB2Individual}}.
 Than, it add the new value calling \noprint{@link:#addDataPropertyB2Individual(OWLNamedIndividual, OWLDataProperty, OWLLiteral, boolean, OWLReferences)}\texttt{\hyperlink{ontologyFramework.OFContextManagement.OWLLibrary.addDataPropertyB2Individual(org.semanticweb.owlapi.model.OWLNamedIndividual,org.semanticweb.owlapi.model.OWLDataProperty,org.semanticweb.owlapi.model.OWLLiteral,boolean,ontologyFramework.OFContextManagement.OWLReferences)}{addDataPropertyB2Individual}}.
 Refer to this last two for how the flag bufferized is used.
\begin{description}
\item[Parameter] ~
\begin{description}
\item[ind]
individual for which a data property will be replaced.
\item[prop]
property to replace
\item[oldValue]
set of old values to remove
\item[newValue]
new value to add
\item[buffered]
flag to buffering changes internally to this class.
\item[ontoRef]
reference to an OWL ontology.
\end{description}
\end{description}
\item[{\ltdHypertarget{ontologyFramework.OFContextManagement.OWLLibrary.replaceDataProperty(org.semanticweb.owlapi.model.OWLNamedIndividual,org.semanticweb.owlapi.model.OWLDataProperty,org.semanticweb.owlapi.model.OWLLiteral,org.semanticweb.owlapi.model.OWLLiteral,java.lang.Boolean,ontologyFramework.OFContextManagement.OWLReferences)}{replaceDataProperty}\label{ontologyFramework.OFContextManagement.OWLLibrary.replaceDataProperty(org.semanticweb.owlapi.model.OWLNamedIndividual,org.semanticweb.owlapi.model.OWLDataProperty,org.semanticweb.owlapi.model.OWLLiteral,org.semanticweb.owlapi.model.OWLLiteral,java.lang.Boolean,ontologyFramework.OFContextManagement.OWLReferences)}}]
~ Atimically (with respect to reasoner update) replacing of a data property.
 Indeed, it will remove the possible data property with a given value
 using \noprint{@link:#removeDataPropertyB2Individual(OWLNamedIndividual, OWLDataProperty, OWLLiteral, boolean, OWLReferences)}\texttt{\hyperlink{ontologyFramework.OFContextManagement.OWLLibrary.removeDataPropertyB2Individual(org.semanticweb.owlapi.model.OWLNamedIndividual,org.semanticweb.owlapi.model.OWLDataProperty,org.semanticweb.owlapi.model.OWLLiteral,boolean,ontologyFramework.OFContextManagement.OWLReferences)}{removeDataPropertyB2Individual}}.
 Than, it add the new value calling \noprint{@link:#addDataPropertyB2Individual(OWLNamedIndividual, OWLDataProperty, OWLLiteral, boolean, OWLReferences)}\texttt{\hyperlink{ontologyFramework.OFContextManagement.OWLLibrary.addDataPropertyB2Individual(org.semanticweb.owlapi.model.OWLNamedIndividual,org.semanticweb.owlapi.model.OWLDataProperty,org.semanticweb.owlapi.model.OWLLiteral,boolean,ontologyFramework.OFContextManagement.OWLReferences)}{addDataPropertyB2Individual}}.
 Refer to this last two for how the flag bufferized is used.
\begin{description}
\item[Parameter] ~
\begin{description}
\item[ind]
individual for which a data property will be replaced.
\item[prop]
property to replace
\item[oldValue]
value to remove
\item[newValue]
new value to add
\item[buffered]
flag to buffering changes internally to this class.
\item[ontoRef]
reference to an OWL ontology.
\end{description}
\end{description}
\item[{\ltdHypertarget{ontologyFramework.OFContextManagement.OWLLibrary.replaceObjectProperty(org.semanticweb.owlapi.model.OWLNamedIndividual,org.semanticweb.owlapi.model.OWLObjectProperty,org.semanticweb.owlapi.model.OWLNamedIndividual,org.semanticweb.owlapi.model.OWLNamedIndividual,java.lang.Boolean,ontologyFramework.OFContextManagement.OWLReferences)}{replaceObjectProperty}\label{ontologyFramework.OFContextManagement.OWLLibrary.replaceObjectProperty(org.semanticweb.owlapi.model.OWLNamedIndividual,org.semanticweb.owlapi.model.OWLObjectProperty,org.semanticweb.owlapi.model.OWLNamedIndividual,org.semanticweb.owlapi.model.OWLNamedIndividual,java.lang.Boolean,ontologyFramework.OFContextManagement.OWLReferences)}}]
~ Atomically (with respect to reasoner update) replacing of a object property.
 Indeed, it will remove the possible object property with a given values
 using \noprint{@link:#removeObjectPropertyB2Individual(OWLNamedIndividual, OWLObjectProperty, OWLNamedIndividual, boolean, OWLReferences)}\texttt{\hyperlink{ontologyFramework.OFContextManagement.OWLLibrary.removeObjectPropertyB2Individual(org.semanticweb.owlapi.model.OWLNamedIndividual,org.semanticweb.owlapi.model.OWLObjectProperty,org.semanticweb.owlapi.model.OWLNamedIndividual,boolean,ontologyFramework.OFContextManagement.OWLReferences)}{removeObjectPropertyB2Individual}}.
 Than, it add the new value calling \noprint{@link:#addObjectPropertyB2Individual(OWLNamedIndividual, OWLObjectProperty, OWLNamedIndividual, boolean, OWLReferences)}\texttt{\hyperlink{ontologyFramework.OFContextManagement.OWLLibrary.addObjectPropertyB2Individual(org.semanticweb.owlapi.model.OWLNamedIndividual,org.semanticweb.owlapi.model.OWLObjectProperty,org.semanticweb.owlapi.model.OWLNamedIndividual,boolean,ontologyFramework.OFContextManagement.OWLReferences)}{addObjectPropertyB2Individual}}.
 Refer to this last two for how the flag bufferized is used.
\begin{description}
\item[Parameter] ~
\begin{description}
\item[ind]
individual for which a object property will be replaced.
\item[prop]
property to replace
\item[oldValue]
set of old values to remove
\item[newValue]
new value to add
\item[buffered]
flag to buffering changes internally to this class.
\item[ontoRef]
reference to an OWL ontology.
\end{description}
\end{description}
\item[{\ltdHypertarget{ontologyFramework.OFContextManagement.OWLLibrary.replaceIndividualClass(org.semanticweb.owlapi.model.OWLNamedIndividual,org.semanticweb.owlapi.model.OWLClass,org.semanticweb.owlapi.model.OWLClass,java.lang.Boolean,ontologyFramework.OFContextManagement.OWLReferences)}{replaceIndividualClass}\label{ontologyFramework.OFContextManagement.OWLLibrary.replaceIndividualClass(org.semanticweb.owlapi.model.OWLNamedIndividual,org.semanticweb.owlapi.model.OWLClass,org.semanticweb.owlapi.model.OWLClass,java.lang.Boolean,ontologyFramework.OFContextManagement.OWLReferences)}}]
~ Atomically (with respect to reasoner update) replacing of individual
 type. Which means to remove an individual from a class and add it to
 belong to another calss.
 Indeed, it will remove the possible type with a given values
 using \noprint{@link:#removeIndividualB2Class(OWLNamedIndividual, OWLClass, boolean, OWLReferences)}\texttt{\hyperlink{ontologyFramework.OFContextManagement.OWLLibrary.removeIndividualB2Class(org.semanticweb.owlapi.model.OWLNamedIndividual,org.semanticweb.owlapi.model.OWLClass,boolean,ontologyFramework.OFContextManagement.OWLReferences)}{removeIndividualB2Class}}.
 Than, it add the new value calling \noprint{@link:#addIndividualB2Class(OWLNamedIndividual, OWLClass, boolean, OWLReferences)}\texttt{\hyperlink{ontologyFramework.OFContextManagement.OWLLibrary.addIndividualB2Class(org.semanticweb.owlapi.model.OWLNamedIndividual,org.semanticweb.owlapi.model.OWLClass,boolean,ontologyFramework.OFContextManagement.OWLReferences)}{addIndividualB2Class}}.
 Refer to this last two for how the flag bufferized is used.
\begin{description}
\item[Parameter] ~
\begin{description}
\item[ind]
individual to change its classification.
\item[oldValue]
old class in which the individual is belonging to
\item[newValue]
new class in which the individual will belonging to
\item[buffered]
flag to buffering changes internally to this class.
\item[ontoRef]
reference to an OWL ontology.
\end{description}
\end{description}
\item[{\ltdHypertarget{ontologyFramework.OFContextManagement.OWLLibrary.getPelletExplanation(ontologyFramework.OFContextManagement.OWLReferences)}{getPelletExplanation}\label{ontologyFramework.OFContextManagement.OWLLibrary.getPelletExplanation(ontologyFramework.OFContextManagement.OWLReferences)}}]
~ It uses Manchester syntax to explain possible inconsistencies.
\begin{description}
\item[Parameter] ~
\begin{description}
\item[ontoRef]
reference to an OWL ontology.
\end{description}
\item[Rückgabewert] 
an inconcistency explanation as a string of text.
\end{description}
\end{description}
